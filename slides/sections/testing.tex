\section{Testing}

\begin{frame}
\frametitle{Pruebas}
\framesubtitle{¿Por qué hacemos pruebas?}

El propósito de las pruebas es incrementar la confianza en el proceso de traducción.

\bigskip

El código generado a partir de la semántica se comporta como sigue:
\begin{itemize}
\item{Abortará su ejecución si encuentra un error de memoria (no se hace promesa alguna acerca de estos programas)}
\item{Producirá un resultado equivalente al del mismo programa interpretado en Isabelle}
\end{itemize}


\end{frame}


\begin{frame}
\frametitle{Pruebas}
\framesubtitle{Tipos de verificaciones}

Se definen tres tipos de verificaciones en la batería de pruebas:

\begin{itemize}
\item{Programas de ejemplo en Chloe}
\item{Pruebas intencionalmente incorrectas}
\item{Programas con pruebas generadas automáticamente que verifican la igualdad de estados finales}
\end{itemize}


\end{frame}


\begin{frame}
\frametitle{Pruebas}
\framesubtitle{Igualdad de estados finales}

Se provee la opción de generar código para probar los programas.

Se quiere verificar que el estado final producido por el programa en C es el mismo que al interpretar el programa.

\begin{block}{Igualdad entre el contenido de las variables globales y la memoria alcanzable.}
\begin{itemize}
\item{Valores enteros}
\item{Apuntadores a Null}
\item{Apuntadores diferentes de Null}
\end{itemize}
\end{block}


\end{frame}


\begin{frame}
\frametitle{Pruebas}
\framesubtitle{Siguiendo apuntadores}

Cuando el contenido de una variable global es un apuntador a memoria, se verifica el contenido completo del bloque de memoria ya que es alcanzable.
\bigskip
\pause

Cuando se encuentra un apuntador a memoria se sigue el mismo hasta que:
\begin{itemize}
\item{Se encuentra un entero o un apuntador a null}
\item{Se encuentra un apuntador que ya fue seguido}
\end{itemize}
\bigskip

\end{frame}


\begin{frame}
\frametitle{Pruebas}
\framesubtitle{Siguiendo los apuntadores en orden DFS}

Para hacer esto se debe:
\begin{itemize}
\item{Seguir los apuntadores en el orden de busqueda en profundidad}
\item{Mantener un conjunto de bloques visitados}
\end{itemize}


Para evitar seguir apuntadores indefinidamente cuando existen referencias cíclicas:

Si se encuentra un apuntador a un bloque de memoria que ya fue visitado, se deja de seguir el apuntador y se comparan los valores de los apuntadores.

\end{frame}


\begin{frame}
\frametitle{Pruebas}
\framesubtitle{Automatización del proceso de pruebas}

Para automatizar el proceso de pruebas se definieron dos ambientes de pruebas (uno en Isabelle y otro en C) que ayudaran en el proceso de generación y corrida de pruebas.
\bigskip
\pause

Además de la generación de pruebas se necesita una manera de ejecutar todas las pruebas en la batería.
\pause
\bigskip

\begin{itemize}
\item{Se define un archivo en Isabelle que importa cada prueba en la batería.}
\item{Se define un Makefile y un shell script que automatizará el proceso de compilación y corrida del código generado.}
\end{itemize}

\end{frame}
