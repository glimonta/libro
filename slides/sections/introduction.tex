\section{Introduction}

\begin{frame}
\frametitle{Motivación}
\framesubtitle{El lenguaje de programación C}

\pause

Lenguaje C:

\pause

\begin{itemize}
\item{Cercanía  a la máquina y bajo \textit{overhead} permiten eficiencia.}
\pause
\item{Utilizado para sistemas operativos, aplicaciones de sistemas embebidos, compiladores, librerías e interpretadores.}
\pause
\end{itemize}

Desventaja?
\pause
\begin{itemize}
\item{Parte de la semántica se define en lenguaje natural, lo cual la hace vulnerable a ambigüedades.}
\end{itemize}

\end{frame}


\begin{frame}
\frametitle{Motivación}
\framesubtitle{Objetivos del trabajo}

\pause
\begin{itemize}
\item{Formalizar la semántica de un lenguaje imperativo que represente un subconjunto determinístico de la semántica de C.}
\pause
\item{Escribir un interpretador dentro del ambiente de Isabelle/HOL y demostrar su correctitud.}
\pause
\item{Generar código C a partir de programas escritos en la semántica formal.}
\pause
\item{Crear un ambiente de pruebas y una batería de pruebas que incrementen la confianza en el proceso de generación de código.}
\end{itemize}

\end{frame}


\begin{frame}
\frametitle{Chloe}
\framesubtitle{Un lenguaje imperativo, subconjunto de C}

\begin{columns}[t]
\column{.45\textwidth}
\begin{block}{Características:}
\pause
\begin{itemize}
\item{Variables}
\pause
\item{Arreglos}
\pause
\item{Aritmética de apuntadores}
\pause
\item{Ciclos}
\pause
\item{Condicionales}
\pause
\item{Funciones}
\pause
\item{Memoria dinámica}
\pause
\end{itemize}
\end{block}
\column{.45\textwidth}
\begin{block}{Limitaciones:}
\begin{itemize}
\pause
\item{Sistema de tipos estático correcto y completo}
\pause
\item{Concurrencia}
\pause
\item{Operaciones I/O}
\pause
\item{Goto}
\pause
\item{Etiquetas}
\pause
\item{Instrucciones break y continue}
\end{itemize}
\pause
\end{block}
\end{columns}

\end{frame}


\begin{frame}
\frametitle{Semántica}

\pause

Existen tres enfoques principales:

\pause

\begin{itemize}
\item{Operacional}
\pause
  \begin{itemize}
    \item{Pasos largos}
    \item{Pasos cortos}
  \end{itemize}
\pause
\item{Denotacional}
\pause
\item{Axiomática}
\pause
\end{itemize}

\end{frame}

\begin{frame}
\frametitle{Isabelle/HOL}

\includegraphics[scale=0.5]{images/isabelle.png}

\begin{itemize}
\item{Isabelle/HOL es un demostrador interactivo de teoremas escrito en ML.}
\pause
\item{Desarrollado por Larry Paulson y Tobias Nipkow.}
\pause
\item{Utiliza el lenguaje HOL para realizar las pruebas.}
\pause
\item{Permite hacer definiciones y demostrar propiedades acerca de las mismas.}
\pause
\item{Se usa la máquina para asistir en las demostraciones}
\pause
\end{itemize}

\end{frame}

\begin{frame}
\frametitle{Ejemplo de una prueba en Isabelle/HOL}

\begin{semiverbatim}
datatype \textit{nat} = 0 | Suc \textit{nat}


fun add :: ``nat => nat => nat'' where

\alert<2>{``add 0 n = n''} |

\alert<4>{``add (Suc m) n = Suc (add m n)''}
\end{semiverbatim}

\begin{columns}[t]
\column{.45\textwidth}
\begin{semiverbatim}
\only<1,2,3,4,5>{
lemma add\_02:

``add m 0 = m''

proof(induction)}
\only<2,3,4,5>{
\alert<2>{

apply simp}
}
\only<4,5>{
\alert<4>{

apply simp}

done
}
\end{semiverbatim}
\column{.45\textwidth}
\begin{block}{Output de Isabelle}
\begin{semiverbatim}
\only<1,2>{
\alert<2>{
1. add 0 0 = 0}

}

\only<1,2,3,4>{
\alert<4>{
2. $\bigwedge$ m. add m 0 = m $\Longrightarrow$

add (Suc m) 0 = Suc m
}}
\only<5>{

No subgoals!
}
\end{semiverbatim}
\end{block}
\end{columns}

\only<2>{
\bigskip

Usando la primera regla de add}
\only<4>{
\bigskip

Usando la segunda regla de add y luego la hipotesis inductiva}

\end{frame}

\begin{comment}

\end{comment}
