\section{Pretty Printing}

\begin{frame}
\frametitle{Pretty Printing}
\framesubtitle{}

\begin{itemize}
\item{Damos una representación en forma de string}
\item{Traducimos el código en Chloe a código en C}
\item{El proceso de traducción es inductivo}
\end{itemize}

\end{frame}

\begin{frame}[fragile]
\frametitle{Pretty Printing}
\framesubtitle{Generación de código para factorial}
\Fontvi

Se toma el siguiente programa en la semántica:

\begin{columns}[t]
\column{.45\textwidth}
\begin{semiverbatim}
definition factorial_decl :: fun_decl
  where "factorial_decl ==
    ( fun_decl.name = fact,
      fun_decl.params = [n],
      fun_decl.locals = [r, i],
      fun_decl.body =
        r ::= (Const 1);;
        i ::= (Const 1);;
        (WHILE (Less (V i) (Plus (V n) (Const 1))) DO
          (r ::= (Mult (V r) (V i));;
          i ::= (Plus (V i) (Const 1)))
        );;
        RETURN (V r)
    )"

definition main_decl :: fun_decl
  where "main_decl ==
    ( fun_decl.name = main,
      fun_decl.params = [],
      fun_decl.locals = [],
      fun_decl.body =
        n ::= Const 5;;
        r ::= fact ([V n])
    )"
\end{semiverbatim}
\column{.45\textwidth}
\begin{semiverbatim}
definition p :: program
  where "p ==
    ( program.name = fact,
      program.globals = [n, r],
      program.procs = [factorial_decl, main_decl]
    )"
\end{semiverbatim}

y se exporta código para el mismo.
\end{columns}


\end{frame}


\begin{frame}[fragile]
\frametitle{Pretty Printing}
\framesubtitle{Generación de código para factorial}
\Fontvi

Se obtiene el siguiente código:

\begin{columns}[t]
\column{.65\textwidth}
\begin{semiverbatim}
\alert<2>{#include <stdlib.h>}
\alert<2>{#include <stdio.h>}
\alert<2>{#include <limits.h>}
\alert<2>{#include <stdint.h>}
\alert<2>{#include "../test_harness.h"}
\alert<2>{#include "../malloc_lib.h"}
\alert<3>{#ifndef INTPTR_MIN}
  \alert<3>{#error ("Macro INTPTR_MIN undefined")}
\alert<3>{#endif}
\alert<3>{#ifndef INTPTR_MAX}
  \alert<3>{#error ("Macro INTPTR_MAX undefined")}
\alert<3>{#endif}
\alert<3>{#if ( INTPTR_MIN + 1 != -9223372036854775807 )}
  \alert<3>{#error (" Assertion INTPTR_MIN + 1 == -9223372036854775807 failed")}
\alert<3>{#endif}
\alert<3>{#if ( INTPTR_MAX != 9223372036854775807 )}
  \alert<3>{#error (" Assertion INTPTR_MAX == 9223372036854775807 failed")}
\alert<3>{#endif}


intptr_t n;
intptr_t r;
\end{semiverbatim}
\column{.35\textwidth}
\begin{semiverbatim}
intptr_t fact(intptr_t n) {
  intptr_t r;
  intptr_t i;
  r = (1);
  i = (1);
  while ((i) < ((n) + (1))) {
    r = ((r) * (i));
    i = ((i) + (1));
  }
  return(r);
}

intptr_t main() {
  n = (5);
  (r) = (fact(n));
}
\end{semiverbatim}
\end{columns}


\end{frame}


\begin{frame}
\frametitle{Pretty Printing}
\framesubtitle{\textit{Casting}}

Se traducen los programas usando el tipo intptr\_t de C.
\bigskip

Este tipo permite que tanto un apuntador como un entero se guarde en el.

Todas las variables se definen con el tipo ``intptr\_t''.

\bigskip

Se debe poder imprimir un \textit{cast} desde y hacia apuntadores para indicar a C como interpretar los valores

\bigskip
\pause

\begin{block}{Convertir a apuntador}
(intptr\_t *) $<$expression$>$

*((intptr\_t *) foo);
\end{block}

\begin{block}{Convertir desde apuntador}
(intptr\_t) $<$expression$>$

(intptr\_t) \_\_MALLOC(sizeof(intptr\_t) * 5);
\end{block}


\end{frame}


\begin{frame}
\frametitle{Pretty Printing}
\framesubtitle{¿Cómo se traduce malloc bajo la suposición de memoria ilimitada?}

Se traduce malloc envolviendola en otra función.

\begin{block}{Generación de código para malloc}
New (Const $9$) $\equiv$ \_\_MALLOC(sizeof(intptr\_t) * $9$)
\end{block}

\bigskip

Esta función abortará la ejecución del programa si encuentra un error de memoria.


\end{frame}


\begin{comment}
\begin{frame}
\frametitle{Exportando Código C}
\framesubtitle{Exportar a un archivo}

Se provee con un mecanismo que permite exportar el programa generado en código C a un archivo en un directorio determinado.


\end{frame}
\end{comment}
