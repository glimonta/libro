\chapter{Pretty Printer}\label{chapter:pretty}
\lhead{Capítulo 4. \emph{Pretty Printer}}

En este capítulo se describe el proceso de \textit{pretty printing} (traducción) que ocurre en la semántica que permite exportar código C.
Se utiliza la implementación de la clase Show de Haskell en Isabelle/HOL hecha por Sternagel y Thiemann~\citeyearpar{Show-AFP} como ayuda en el proceso de traducción.
Sternagel y Thiemann implementan un \textit{type class} para funciones \verb|to-string| así como instanciaciones para los tipos estándar de Isabelle/HOL.
Ademas permiten la derivación de funciones show para tipos de datos definidos por el usuario.
Se instancia esta clase al crear una función \verb|show| para cada uno de los tipos de datos que se definen progresivamente hasta que se torna posible imprimir un programa.
En el apéndice~\ref{ap:pretty_printing} se explica en detalle las cadenas de caracteres concretas que se obtienen como resultado de la traducción.
Esta sección se limitará a comentar los aspectos más relevantes del proceso de traducción.
\begin{comment}
Las siguientes secciones explicarán en detalle las cadenas de caracteres concretos que se obtienen como resultado de la traducción.
\end{comment}

\begin{comment}
\section{Palabras}\label{section:pretty_words}
La primera instanciación de \verb|shows| que se debe hacer es aquella para las palabras de modo que sea posible imprimir valores de este tipo.
Para hacer \textit{pretty printing} de un valor del tipo palabra simplemente se hace \textit{casting} de ese valor a un tipo entero predefinido por Isabelle/HOL y se utiliza la función \verb|show| para ese tipo.
Como resultado, las palabras serán impresas como enteros con signo.

\section{Valores}\label{section:pretty_values}

Aunque el tipo de datos \verb|val| y las valuaciones no son utilizadas en el proceso de generación de código, se considera útil tener un mecanismo para hacer \textit{pretty printing} de los mismos con fines de depuración.

\subsection{Tipo val}\label{subsection:pretty_val_type}
El tipo \verb|val| es el primer tipo de datos definido por el usuario para el que se presenta una instanciación.
En la tabla~\ref{tab:pretty_val} se encuentra la equivalencia entre la sintaxis abstracta para el tipo \verb|val| y la representación en cadena de caracteres.
Nótese que $w$, $base$ y $ofs$ representan palabras, naturales y enteros, por lo tanto sus funciones \verb|show| serán utilizadas para obtener su representación en cadena de caracteres, por ejemplo, la representación en cadena de caracteres de $\mathtt{I}\ 42$ sería $"42"$, la representación en cadena de caracteres de $\mathtt{A}\ (4,\ 2)$ sería $"4[2]"$.
Se puede realizar \verb|show| de una lista de valores al hacer \verb|show| de cada valor en una notación de listas, es decir, $[vname\ =\ valor, \dots, vname_n\ =\ valor_n]$.

Se rodean con cuñas los parámetros ($"<\ >"$) para indicar que lo que está dentro de ellos es una representación en forma de cadena de caracteres resultante de aplicar una función \verb|show| en el parámetro y se continuará usando esta notación a lo largo del documento.

\begin{table}[h!]
\centering
\begin{tabular}{|c|c|}
  \hline
  \textbf{Sintaxis abstracta} & \textbf{Representación en cadena de caracteres} \\ [0.5ex]
  \hline \hline
  \verb|NullVal| & null \\
  \verb|I| $w$ & $<w>$ \\
  \verb|A| $(base,\ ofs)$ & $<base>$[$<ofs>$] \\
  \hline
\end{tabular}

\caption{Traducción del tipo val}
\label{tab:pretty_val}
\end{table}

\subsection{Tipo val option}\label{subsection:pretty_val_option_type}

Un \verb|val option| tiene el significado semántico de un valor inicializado y no inicializado.
La tabla~\ref{tab:pretty_val_option} muestra la equivalencia entre la sintaxis abstracta y la representación en cadena de caracteres.
Un valor inicializado simplemente será la cadena de caracteres resultante de llamar a la función \verb|show| sobre ese valor, mientras que la representación de un valor no inicializado será un $''?''$ para representar que el valor es aun desconocido.

\begin{table}[h!]
\centering
\begin{tabular}{|c|c|}
  \hline
  \textbf{Sintaxis abstracta} & \textbf{Representación en cadena de caracteres} \\ [0.5ex]
  \hline \hline
  \verb|Some| v & $<v>$ \\
  \verb|None| & ? \\
  \hline
\end{tabular}

\caption{Traducción del tipo val option}
\label{tab:pretty_val_option}
\end{table}


\subsection{Valuaciones}\label{subsection:pretty_valuations}

Para representar una valuación se necesita un parámetro extra: una lista de nombres de variables, esta lista tendrá los nombres de las variables para los cuales se quiere imprimir su valor en la valuación.
La valuación será impresa con el siguiente formato:

\begin{equation*}
[<vname_0> = <valor_0>, <vname_1> = <valor_1>, \dots, <vname_n> = <valor_n>]
\end{equation*}

Por ejemplo, si se toma la valuación $[foo \mapsto \mathtt{Some}(\mathtt{I}\ 15), bar \mapsto None, baz \mapsto \mathtt{Some}(\mathtt{A}(1,9))]$ y la lista de variables $[foo, bar, baz]$ entonces se obtiene la siguiente representación para la valuación: $[foo\ =\ 15,\ bar\ =\ ?,\ baz\ =\ 1[9]]$.


\section{Memoria}\label{section:pretty_memory}

La memoria, al igual que los valores, no es un componente necesario para generar programas en C.
No obstante, tener una manera de hacer \textit{pretty printing} de la memoria en cierto estado es útil al de realizar depuración.
Para imprimir la memoria se debe saber como hacer \verb|show| del contenido de un bloque y de todo el bloque.

Cuando se accede a un bloque a memoria se puede obtener tanto el contenido del bloque o un valor \verb|None| que indica un bloque libre en memoria.
La tabla~\ref{tab:pretty_block_content} muestra la representación en cadena de caracteres de esto.
Nótese que en el caso de un valor \verb|None| se imprime la cadena \verb|free| rodeada de cuñas, siendo las mismas parte de la representación y una excepción a la notación descrita anteriormente.
Si el contenido del bloque es una lista de valores entonces se imprime la lista de valores.

\begin{table}[h!]
\centering
\begin{tabular}{|c|c|}
  \hline
  \textbf{Sintaxis abstracta} & \textbf{Representación en cadena de caracteres} \\ [0.5ex]
  \hline \hline
  \verb|Some| content & $<content>$ \\
  \verb|None| & $<$free$>$ \\
  \hline
\end{tabular}

\caption{Traducción del contenido de un bloque}
\label{tab:pretty_block_content}
\end{table}

Un bloque completo se imprime de la siguiente manera:
\begin{equation*}
<base>\ :\ <contenido\_bloque> 
\end{equation*}

donde $base$ es el primer componente de una dirección con la que se accede al bloque y $contenido\_bloque$ es la representación en cadena de caracteres del contenido en el bloque número $base$.

Para imprimir la memoria completa se hace \verb|show| de cada bloque existente en memoria.
Por ejemplo, la representación en cadena de caracteres del estado de memoria: $[\mathtt{Some}\ [\mathtt{I}\ 13,\ \mathtt{None}],\ None,\ \mathtt{Some}\ [\mathtt{A}(2,3),\ None, \mathtt{I}\ 56]]$ sería:

\begin{align*}
0\ &:\ [13,\ ?] \\
1\ &:\ <free> \\
2\ &:\ [2[3],\ ?,\ 56]
\end{align*}
\end{comment}


\section{Expresiones}

Para imprimir expresiones se deben poder imprimir operaciones unarias y binarias (las cuales se encuentran detalladas en el apéndice~\ref{ap:pretty_printing}).
También se debe poder imprimir valores a los que se les ha hecho \textit{casting} desde y hacia apuntadores.
Se utiliza el tipo \verb|intptr_t| de C, el cual es lo suficientemente grande como para guardar el valor de un entero así como el de un apuntador.
Esto se hace debido a que solo se tienen valores enteros en el lenguaje y las direcciones y los valores enteros son disjuntos al nivel de semántica.

En el nivel de lenguaje C se le debe poder indicar al compilador cuando un valor debe ser interpretado como una dirección y hacerle \textit{cast} como tal.
Se permite este \textit{casting} entre direcciones y enteros durante el proceso de traducción dado que se conoce con seguridad cuando un valor debe ser interpretado como una dirección y cuando debe ser interpretado como un entero, mientras que C no.

\begin{comment}
\paragraph*{Operaciones unarias y binarias}
Al hacer \textit{pretty printing} de operaciones binarias se utilizan paréntesis alrededor de cada expresión que se imprime.
Esto, naturalmente, generará más paréntesis de los necesarios pero se toma esta decisión para asegurar que el orden de evaluación se mantenga igual al previsto y que no se obtengan diferentes órdenes de evaluación debido a la precedencia de los operadores.

Un operador binario se imprime usando notación de infijo: $(<operando_1> <operador> <operando_2>)$.
Ejemplos de esto se muestran en la tabla~\ref{tab:pretty_bin_op}.

\begin{table}[h!]
\centering
\begin{tabular}{|c|c|}
  \hline
  \textbf{Sintaxis abstracta} & \textbf{Representación en cadena de caracteres} \\ [0.5ex]
  \hline \hline
  \verb|Plus| (\verb|Const| $11$) (\verb|Const| $11$) & ($11\ +\ 11$) \\
  \verb|Subst| (\verb|Const| $9$) (\verb|Const| $5$) & ($9\ -\ 5$) \\
  \verb|Mult| (\verb|Const| $2$) (\verb|Const| $3$) & ($2\ *\ 3$) \\
  \hline
\end{tabular}

\caption{Ejemplos de \textit{pretty printing} de operadores binarios}
\label{tab:pretty_bin_op}
\end{table}

Un operador unario se imprime utilizando notación de prefijo: $<operador> (<operando>)$.
Nótese que se rodea entre paréntesis el operando con el fin de garantizar la correcta precedencia en las operaciones.
Ejemplos de esto se muestran en la tabla~\ref{tab:pretty_un_op}.

\begin{table}[h!]
\centering
\begin{tabular}{|c|c|}
  \hline
  \textbf{Sintaxis abstracta} & \textbf{Representación en cadena de caracteres} \\ [0.5ex]
  \hline \hline
  \verb|Minus| (\verb|Const| $11$) & $-\ (11)$ \\
  \verb|Not| (\verb|Const| $0$) & $!\ (0)$ \\
  \hline
\end{tabular}

\caption{Ejemplos de \textit{pretty printing} de operadores unarios}
\label{tab:pretty_un_op}
\end{table}
\end{comment}

\paragraph*{\textit{Casts}}
Dado que los valores con los que se trabaja deben ser interpretados en C algunas veces como enteros y otras como apuntadores, se debe poder imprimir un \textit{cast} explícito entre esos dos tipos en el programa generado.
Se incluyen \textit{casts} a apuntadores cuando se trata con operaciones de referencia, desreferencia y acceso a arreglos.
Se quiere hacer \textit{casting} a entero en el caso de asignación de memoria.
Una asignación de memoria retorna un apuntador, pero para poder asignarlo a una variable se le debe hacer \textit{casting} como un entero.
Dado que todas las variables del programa se declaran con el tipo \verb|intptr_t| y no \verb|intptr_t *|, se debe hacer esto, además se sabe que al trabajar con direcciones las mismas serán interpretadas de la manera correcta ya que se agregan \textit{casts} a apuntadores.


Un \textit{cast} a una dirección se imprime de la siguiente manera: $(\mathtt{intptr\_t}\ *) <expresion>$.
Un \textit{cast} a un valor entero se imprime de la siguiente manera: $(\mathtt{intptr\_t}) <expresion>$.
En la tabla~\ref{tab:pretty_casts} se encuentran ejemplos del \textit{pretty printing} de \textit{casts}.

\begin{table}[h!]
\centering
\begin{tabular}{|c|c|}
  \hline
  \textbf{Sintaxis abstracta} & \textbf{Representación en cadena de caracteres} \\ [0.5ex]
  \hline \hline
  \verb|Deref| (\verb|V| $foo$) & $*((\mathtt{intptr\_t}\ *)\ $foo$)$ \\
  \verb|Ref| (\verb|V| $foo$)   & $((\mathtt{intptr\_t}\ *)\ \&($foo$))$ \\
  \verb|New| (\verb|Const| $9$) & $(\mathtt{intptr\_t})\ \mathtt{\_\_MALLOC}(\mathtt{sizeof}(\mathtt{intptr\_t)}\ *\ ($9$))$ \\
  \hline
\end{tabular}

\caption{Ejemplos de \textit{pretty printing} para \textit{casts}}
\label{tab:pretty_casts}
\end{table}


\paragraph*{Asignaciones de memoria}

Como fue mencionado en el capítulo~\ref{chapter:semantics} el comportamiento de la función de asignación de este trabajo y la función de asignación de memoria de C difieren debido a que en este trabajo se supone que la memoria es ilimitada.
Por lo tanto no se puede simplemente traducir la asignación de memoria a una llamada a la función \verb|malloc| en C.

Se debe envolver la llamada a la función \verb|malloc| de C en otra función que aborte el programa en el caso en que exista una falla por falta de memoria.
Se define una función ``\verb|__MALLOC|'' que toma el tamaño del nuevo bloque de memoria que será asignado con \verb|malloc|, hace la llamada a \verb|malloc| y retorna el apuntador al nuevo bloque de memoria en el caso de que la llamada sea exitosa y si la llamada falla entonces aborta el programa con el código de salida 3.
Se define el código de salida 3 como un código de salida erróneo que significa que hubo un error al asignar memoria para poder detectar este error luego en el proceso de pruebas.


\begin{comment}
\paragraph*{Expresiones}
En la tabla~\ref{tab:pretty_expressions} se presenta la representación en cadena de caracteres para cada una de las expresiones.
Se utilizan operaciones simples como operandos tales como variables o valores constantes por simplicidad, sin embargo las expresiones pueden estar compuestas por términos más complejos.

\begin{table}[h!]
\centering
\begin{tabular}{|c|c|}
  \hline
  \textbf{Sintaxis abstracta} & \textbf{Representación en cadena de caracteres} \\ [0.5ex]
  \hline \hline
  \verb|Const| $42$ & $42$ \\
  \verb|Null| & (\verb|intptr_t| *) $0$ \\
  \verb|V| $x$ & $x$ \\
  \verb|Plus| $(\mathtt{Const}\ 2)\ (\mathtt{Const}5)$ & $(2\ +\ 5)$ \\
  \verb|Subst| $(\mathtt{Const}\ 9\ (\mathtt{Const}5)$ & $(9\ -\ 5)$ \\
  \verb|Minus| $(\mathtt{Const}\ 9)$ & ($-9$) \\
  \verb|Div| $(\mathtt{Const}\ 8)\ (\mathtt{Const}\ 4)$ & $(8\ /\ 4)$ \\
  \verb|Mod| $(\mathtt{Const}\ 8)\ (\mathtt{Const}\ 4)$ & $(8\ \%\ 4)$ \\
  \verb|Mult| $(\mathtt{Const}\ 9)\ (\mathtt{Const}\ 3)$ & $(9\ *\ 3)$ \\
  \verb|Less| $(\mathtt{Const}\ 7)\ (\mathtt{Const}\ 9)$ & $(7\ <\ 9)$ \\
  \verb|Not| $(\mathtt{Const}\ 0)$ & $!\ (0)$ \\
  \verb|And| $(\mathtt{Const}\ 1)\ (\mathtt{Const}\ 1)$ & $(1\ \&\&\ 1)$ \\
  \verb|Or| $(\mathtt{Const}\ 1)\ (\mathtt{Const}\ 0)$ & $(1\ || 0)$ \\
  \verb|Eq| $(\mathtt{Const}\ 6)\ (\mathtt{Const}\ 4)$ & $(6\ == 4)$ \\
  \verb|New| $(\mathtt{Const}\ 9)$ & $(\mathtt{intptr\_t})\ \mathtt{\_\_MALLOC}(\mathtt{sizeof}(\mathtt{intptr\_t)}\ *\ ($9$))$ \\
  \verb|Deref| $(\mathtt{V}\ foo)$ & $*((\mathtt{intptr\_t}\ *)\ foo)$ \\
  \verb|Ref| $(\mathtt{V}\ foo)$ & $((\mathtt{intptr\_t}\ *)\ \&(foo))$ \\
  \verb|Index| $(\mathtt{V}\ bar)\ (\mathtt{Const}\ 3)$ & $((\mathtt{intptr\_t}\ *)\ bar[3])$ \\
  \verb|Derefl| $(\mathtt{V}\ foo)$ & $(*((\mathtt{intptr\_t}\ *)\ foo))$ \\
  \verb|Indexl| $(\mathtt{V}\ bar)\ (\mathtt{Const}\ 3)$ & $((\mathtt{intptr\_t}\ *)\ bar[3])$ \\
  \hline
\end{tabular}

\caption{Ejemplos de \textit{pretty printing} para expresiones}
\label{tab:pretty_expressions}
\end{table}


\section{Comandos}

Primeramente, se necesita un mecanismo que permita imprimir comandos indentados para facilitar la generación de código.
Se definen dos abreviaciones auxiliares que imprimen espacios en blanco para indentación al inicio de un término.
La razón por la que se definen dos abreviaciones es porque una de ellas también imprime un terminador $";"$ luego del término, mientras que la otra no.

Las llamadas a funciones se imprimen siguiendo el siguiente formato: $<nombre\_funcion>\ ([<argumento_0, argumento_1, \dots, <argumento_n>])$ donde los corchetes ($[]$) indican que los argumentos son opcionales.

Los comandos en Chloe se imprimen como se muestra en los ejemplos de la tabla~\ref{tab:pretty_commands}.
Se utiliza $''$ para indicar una cadena vacía.
El nivel correcto de indentación se indica en los parámetros de la función que se encarga de realizar el \textit{pretty printing}, acá esos son omitidos y en su lugar se indica donde la indentación aumenta.
La indentación aumenta al imprimir comandos que se encuentren dentro de un bloque, por ejemplo: las ramas de un condicional o el cuerpo de un ciclo.

\begin{table}[h!]
\centering
\begin{tabular}{|c|l|}
  \hline
  \textbf{Sintaxis abstracta} & \textbf{Representación en cadena de caracteres} \\ [0.5ex]
  \hline \hline
  \verb|SKIP| & $''$ \\
  \hline
  $\mathtt{Derefl}\ foo\ ::==\ \mathtt{Const}\ 4$                            & $*((\mathtt{intptr\_t}\ *)\ foo)\ =\ 4;$ \\
  \hline
  $foo\ ::=\ \mathtt{Const}\ 4$                                              & $foo\ =\ 4;$ \\
  \hline
  $c_1;;\ c_2$                                                               & $<c_1> <c_2>$ \\
  \hline
  $\mathtt{IF}\ (\mathtt{V}\ b)\ \mathtt{THEN}$                              & $\mathtt{if}\ (b)\ \{$\\
  $foo\ ::=\ \mathtt{Const}\ 4$                                              & $\ \ foo\ =\ 4;$ \\
  $\mathtt{ELSE}$                                                            &  $\}$ \\
  $\mathtt{SKIP}$                                                            & \\
  \hline
  $\mathtt{If}\ (\mathtt{V}\ b)\ \mathtt{THEN}$                              & $\mathtt{if}\ (b)\ \{$\\
  $foo\ ::=\ \mathtt{Const}\ 4)$                                             & $\ \ foo\ =\ 4;$ \\
  $\mathtt{ELSE}$                                                            & $\}\ else\ \{$ \\
  $bar\ ::=\ \mathtt{Const}\ 3)$                                             & $\ \ bar\ =\ 3;$ \\
                                                                             & $\}$ \\
  \hline
  $\mathtt{While}\ (\mathtt{V}\ b)\ \mathtt{DO}$                             & $\mathtt{while}\ (b)\ \{$\\
  $foo\ ::=\ \mathtt{Const}\ 4$                                              & $\ \ foo\ =\ 4;$ \\
                                                                             & $\}$ \\
  \hline
  \verb|FREE| $(\mathtt{Derefl}\ foo)$                                       & $\mathtt{free}\ (\&\ (*((\mathtt{intptr\_t}\ *)\ foo));$ \\
  \hline
  \verb|RETURN| $(\mathtt{V}\ foo)$                                          & $\mathtt{return}\ (foo));$ \\
  \hline
  \verb|RETURNV|                                                             & $\mathtt{return};$ \\
  \hline
  $(\mathtt{Derefl}\ foo)\ ::==\ bar\ ([\mathtt{V}\ baz,\mathtt{Const}\ 4])$ & $*((\mathtt{intptr\_t}\ *)\ foo)\ =\ bar(baz,\ 4);$ \\
  \hline
  $foo\ ::=\ bar\ ([\mathtt{Const}\ 65])$                                    & $foo =\ bar(65);$ \\
  \hline
  \verb|CALL| $bar\ ([])$                                                    & $bar();$ \\
  \hline
\end{tabular}

\caption{Ejemplos de \textit{pretty printing} para comandos}
\label{tab:pretty_commands}
\end{table}


\section{Declaraciones de funciones}

A continuación se define como se imprimen las declaraciones de funciones.
Para hacerlo, se imprime la definición de una función de acuerdo al siguiente formato:
\begin{equation*}
\begin{split}
&\mathtt{intptr\_t} \ <nombre\_de\_funcion>(\mathtt{intptr\_t}\ <nombre\_arg_0>,\ \dots,\\
&\ \ \ \ \ \ \ \ \ \ \ \ \ \ \ \ \ \ \ \ \ \ \mathtt{intptr\_t}\ <nombre\_arg_n)\ \{ \\
&\ \ \mathtt{intptr\_t}\ <var\_local_0> \\
&\vdots \\
&\ \ \mathtt{intptr\_t}\ <var\_local_n> \\
&<cuerpo> \\
&\} \\
\end{split}
\end{equation*}

El retorno, los argumentos y las variables locales es de tipo \verb|intptr_t| dado que, como se mencionó anteriormente, solo se tiene un tipo en el proceso de traducción y se realizan \textit{casts} hacia y desde apuntadores cuando son necesarios.

Un ejemplo de la traducción de la declaración de una función para una función factorial se encuentra en la tabla~\ref{tab:pretty_function_fact}.
En este ejemplo se evita el uso de $''\ ''$ para representar cadenas de caracteres en Isabelle/HOL y simplemente se escribe la cadena de caracteres sin las comillas.

\begin{table}[h!]
\centering
\begin{tabular}{|l|l|}
  \hline
  \textbf{Sintaxis abstracta} & \textbf{Representación en cadena de caracteres} \\ [0.5ex]
  \hline \hline
  \verb|definition factorial_decl| $::$ \verb|fun_decl|  & \verb|intptr_t fact(intptr_t n) {| \\
  \verb|where "factorial_decl| $\equiv$                  & \verb|  intptr_t r;| \\
  \verb|  ( fun_decl.name = fact,|                       & \verb|  intptr_t i;| \\
  \verb|    fun_decl.params = [n],|                      & \verb|  r = (1);| \\
  \verb|    fun_decl.locals = [r, i],|                   & \verb|  i = (1);| \\
  \verb|    fun_decl.body =|                             & \verb|  while ((i) < ((n) + (1))) {| \\
  \verb|      r ::= (Const 1);;|                         & \verb|    r = ((r) * (i));| \\
  \verb|      i ::= (Const 1);;|                         & \verb|    i = ((i) + (1));| \\
  \verb|      (WHILE|                                    & \verb|  }| \\
  \verb|         (Less (V i) (Plus (V n) (Const 1)))|    & \verb|  return(r);| \\
  \verb|      DO|                                        & \verb|}| \\
  \verb|        (r ::= (Mult (V r) (V i));;|             & \\
  \verb|        i ::= (Plus (V i) (Const 1)))|           & \\
  \verb|      );;|                                       & \\
  \verb|    RETURN (V r)|                                & \\
  \verb|  )"|                                            & \\
  \hline
\end{tabular}

\caption{Ejemplo de \textit{pretty printing} de la declaración de una función factorial}
\label{tab:pretty_function_fact}
\end{table}


\section{Estados}

Al ejecutar un programa dentro del ambiente de Isabelle/HOL a menudo se quieren inspeccionar los estados.
En esta sección se define una forma sencilla de inspeccionar los estados al tener una representación en cadena de caracteres para los mismos.

Primero se describe como se imprime una ubicación de retorno.
Se instancia la clase \verb|show| para el tipo \verb|return_loc|.
En la tabla~\ref{tab:pretty_rloc} se encuentra la equivalencia entre la sintaxis abstracta para el tipo \verb|return_loc| y su representación en cadena de caracteres.
Donde $<base>$ y $<ofs>$ son el resultado de aplicar la función \verb|show| sobre $base$ y $ofs$ y $<invalid>$ es una cadena de caracteres literal que incluye las cuñas.

\begin{table}[h!]
\centering
\begin{tabular}{|c|c|}
  \hline
  \textbf{Sintaxis abstracta} & \textbf{Representación en cadena de caracteres} \\ [0.5ex]
  \hline \hline
  \verb|Ar| $(base,\ ofs)$ & $<base>$[$<ofs>$] \\
  \verb|Vr| $w$ & $w$ \\
  \verb|Invalid| & $<$invalid$>$ \\
  \hline
\end{tabular}

\caption{Traducción del tipo de una ubicación de retorno}
\label{tab:pretty_rloc}
\end{table}

Luego se describe como se imprime la pila.
Un solo marco de pila se imprime al imprimir el comando, la lista de variables locales y la ubicación de retorno esperada con el siguiente formato: $rloc\ =\ <rloc>$ separados por un salto de línea.
Para imprimir la pila completa, se imprime cada marco de pila separado por ``\verb|---------------|''.


Para imprimir un estado se debe dar a la función \verb|show| una lista de nombres de variables, la cual será utilizada para imprimir la valuación para las variables globales y para las locales en cada marco de pila.
Un estado se imprime al imprimir la pila, los valores de las variables globales y, finalmente, la memoria, separados por ``\verb|=================|''
Un ejemplo de \textit{pretty printing} para un estado se muestra en la tabla~\ref{tab:pretty_simp_state}.

\begin{table}[h!]
\centering
\begin{tabular}{|l|l|}
  \hline
  \textbf{Sintaxis abstracta} & \textbf{Representación en cadena de caracteres} \\ [0.5ex]
  \hline \hline
  $(\ [\ (x\ ::=\ \mathtt{Const}\ 4;;\ y\ ::=\ \mathtt{Const}\ 25,$                                            & \verb|x = (4);| \\
  $\ \ \ \ \ \ [z\ \mapsto\ \mathtt{Some}\ (\mathtt{I}\ 0)],\ \mathtt{Invalid}),$                              & \verb|y = (25);| \\
  $\ \ [\ (x\ ::=\ \mathtt{Const}\ 3;;\ y\ ::=\ \mathtt{Const}\ 43,$                                           & \verb|---------------| \\
  $\ \ \ \ \ \ [z\ \mapsto\ \mathtt{Some}\ (\mathtt{I}\ 6)],\ \mathtt{Vr} foo)],$                              & \verb|[z = 0]| \\
  $\ \ [x\ \mapsto\ \mathtt{Some}\ (\mathtt{I}\ 3),\ y\ \mapsto\ \mathtt{Some}\ (\mathtt{I}\ 8),$              & \verb|---------------| \\
  $\ \ foo\ \mapsto\ \mathtt{Some}\ (\mathtt{I}\ 0)],$                                                         & \verb|rloc = <invalid>| \\
  $\ \ [\mathtt{None},\ \mathtt{Some}\ [\mathtt{Some}\ (\mathtt{I}\ 44), \mathtt{Some}\ (\mathtt{A}\ (2,0))],$ & \verb|x = (5);| \\
  $\ \ \mathtt{Some}\ [\mathtt{Some}\ (\mathtt{I}\ 78)]$                                                       & \verb|y = (43);| \\
                                                                                                               & \verb|---------------| \\
                                                                                                               & \verb|[z = 6]| \\
                                                                                                               & \verb|---------------| \\
                                                                                                               & \verb|rloc = foo| \\
                                                                                                               & \verb|=================| \\
                                                                                                               & \verb|[x = 3, y = 8, foo = 0]| \\
                                                                                                               & \verb|=================| \\
                                                                                                               & \verb|1: <free>| \\
                                                                                                               & \verb|2: [44, *2[0]]| \\
                                                                                                               & \verb|3: [78]| \\
  \hline
\end{tabular}

\caption{Ejemplo de \textit{pretty printing} para un estado}
\label{tab:pretty_simp_state}
\end{table}
\end{comment}


\section{Programas}

\begin{comment}
En esta sección se discute la traducción de un programa completo.
\end{comment}

\paragraph*{Archivos de cabecera y chequeo de límites}

En el código C que será exportado se quiere incluir los archivos de cabecera para las librerías estándar de C (\verb|stdlib.h| y \verb|stdio.h|).
Además se incluyen dos archivos de cabecera adicionales, una permite el uso del tipo \verb|intptr_t| y la otra es donde se encuentran los macros con las definiciones de los límites de los tipos enteros.
Estas son \verb|limits.h| y \verb|stdint.h|, respectivamente.
También se desea incluir el archivo de cabecera que define ciertos macros utilizados por el arnés de pruebas con fines de pruebas, este tema será tratado con más detalle en el capítulo~\ref{chapter:testing}.
Por último, se incluye el archivo de cabecera que contiene la función para manejar las llamadas a \verb|malloc|.
Antes de generar alguna otra parte del programa se quiere imprimir las directivas para incluir los archivos de cabecera mencionados anteriormente.


Además, el proceso de traducción genera un programa que será compilable y ejecutable.
Esto es cierto para las arquitecturas que soporten al menos el mismo rango para los tipos enteros que esta abstracción.
Esto significa que, cualquier arquitectura donde el programa sea compilado y ejecutado debe cumplir con las restricciones supuestas por esta semántica para valores enteros.
Por lo tanto, se utiliza el preprocesador de C para imprimir una verificación de límites de enteros que afirma que los macros que contienen los límites superiores e inferiores para el tipo entero se encuentren definidos.
En este caso, se debe verificar que los macros \verb|INTPTR_MIN| y \verb|INTPTR_MAX| estén ambos definidos.
A continuación se verifica que los límites definidos en tales macros son iguales a los límites supuestos en la semántica (los límites supuestos son \verb|INT_MIN| y \verb|INT_MAX| y se definen en la figura~\ref{fig:int_bounds}).


El tipo utilizado para la traducción al igual que la precisión de los enteros puede ser cambiado ya que están parametrizados.
Se definen los valores y se utilizan en la semántica y en el proceso de traducción.
Las definiciones para el tipo utilizado en esta traducción son las siguientes:

\begin{lstlisting}[frame=single, mathescape=true]
definition "dflt_type $\equiv$ ''intptr_t''"
definition "dflt_type_bound_name $\equiv$ ''INTPTR''"

definition "dflt_type_min_bound_name $\equiv$ dflt_type_bound_name $@$ ''_MIN''"
definition "dflt_type_max_bound_name $\equiv$ dflt_type_bound_name $@$ ''_MAX''"
\end{lstlisting}

donde $@$ representa la operación \textit{append} entre cadenas de caracteres.
Nótese que la precisión del tipo utilizado como \verb|dflt_type| debe corresponder a la precisión definida en \verb|int_width|.

\paragraph*{Variables globales}

La impresión de variables globales se hace imprimiendo la lista de variables, siguiendo el formato: $\mathtt{intptr\_t}\ <nombre_variable>;$.

\begin{comment}
\begin{equation*}
\begin{split}
& \mathtt{intptr\_t}\ <variable\_name_0>; \\
& \vdots \\
& \mathtt{intptr\_t}\ <variable\_name_n>; \\
\end{split}
\end{equation*}
\end{comment}

\paragraph*{Programas}

Para imprimir un programa se imprimen las directivas \verb|include| para los archivos de cabecera, luego la verificación de los límites para los enteros, las declaraciones de las variables globales y, por último, se imprime cada función en el programa separado por un carácter de salto de línea.
En el apéndice~\ref{ap:factorial_ejemplo} se encuentra la definición de un programa factorial en Isabelle y el código C generado para el mismo.


\begin{comment}

apéndices

\begin{figure}
\begin{lstlisting}[mathescape=true]
  definition factorial_decl :: fun_decl
    where "factorial_decl $\equiv$
      ( fun_decl.name = fact,
        fun_decl.params = [n],
        fun_decl.locals = [r, i],
        fun_decl.body =
          r ::= (Const 1);;
          i ::= (Const 1);;
          (WHILE (Less (V i) (Plus (V n) (Const 1))) DO
            (r ::= (Mult (V r) (V i));;
            i ::= (Plus (V i) (Const 1)))
          );;
          RETURN (V r)
      )"

  definition main_decl :: fun_decl
    where "main_decl $\equiv$
      ( fun_decl.name = main,
        fun_decl.params = [],
        fun_decl.locals = [],
        fun_decl.body =
          n ::= Const 5;;
          r ::= fact ([V n])
      )"

  definition p :: program
    where "p $\equiv$
      ( program.name = fact,
        program.globals = [n, r],
        program.procs = [factorial_decl, main_decl]
      )"
\end{lstlisting}
\caption{Definición de factorial en Isabelle}
\label{fig:factorial_isabelle}
\end{figure}


\begin{figure}
\begin{lstlisting}[mathescape=true]
  #include <stdlib.h>
  #include <stdio.h>
  #include <limits.h>
  #include <stdint.h>
  #include "../test_harness.h"
  #include "../malloc_lib.h"
  #ifndef INTPTR_MIN
    #error ("Macro INTPTR_MIN undefined")
  #endif
  #ifndef INTPTR_MAX
    #error ("Macro INTPTR_MAX undefined")
  #endif
  #if ( INTPTR_MIN + 1 != -9223372036854775807 )
    #error ("Assertion INTPTR_MIN + 1 == -9223372036854775807 failed")
  #endif
  #if ( INTPTR_MAX != 9223372036854775807 )
    #error ("Assertion INTPTR_MAX == 9223372036854775807 failed")
  #endif


  intptr_t n;
  intptr_t r;

  intptr_t fact(intptr_t n) {
    intptr_t r;
    intptr_t i;
    r = (1);
    i = (1);
    while ((i) < ((n) + (1))) {
      r = ((r) * (i));
      i = ((i) + (1));
    }
    return(r);
  }

  intptr_t main() {
    n = (5);
    (r) = (fact(n));
  }
\end{lstlisting}
\caption{Programa traducido a C}
\label{fig:factorial_c}
\end{figure}

\end{comment}

\section{Exportando código C}\label{section:exporting_c_code}

Ahora que hay una manera de traducir un programa completo desde la semántica a código C, se puede exportar el código generado para C.


\begin{comment}
\begin{lstlisting}[mathescape=true]
  definition prepare_export :: "program $\Rightarrow$ string option" where
    "prepare_export prog $\equiv$ do {
      assert (valid_program prog);
      Some (shows_prog prog '''')
    }"
\end{lstlisting}
\end{comment}

Solo se exporta código para programas válidos.
Para garantizar esto se define una función que prepara un programa para su exportación.
Esta función toma un programa y verifica que sea válido, de acuerdo a la definición de validez dada en~\ref{section:programs_commands}.
Si no es válido, se retorna un valor \verb|None|, en caso contrario se genera la cadena de caracteres que contiene el programa en C con la función \verb|shows_prog|.~\footnote{No se incluye el código para las funciones show en este documento sino que se explica como funcionan. Los archivos de Isabelle pueden ser chequeados para detalles de implementación.}

Se exporta código C a archivos externos.
El nombre del archivo está dado por el nombre del programa en su definición.
Se define una función en ML dentro de Isabelle que exporta el código C de un programa.
La definición de la misma se encuentra en el apéndice~\ref{ap:generate_c_code}.


\begin{comment}
\begin{lstlisting}
  fun export_c_code (SOME code) rel_path name thy =
    let
      val str = code |> String.implode;
    in
      if rel_path="" orelse name="" then
        (writeln str; thy)
      else let
        val base_path = Resources.master_directory thy
        val rel_path = Path.explode rel_path
        val name_path = Path.basic name |> Path.ext "c"

        val abs_path = Path.appends [base_path, rel_path, name_path]
        val abs_path = Path.implode abs_path

        val _ = writeln ("Writing to file " ^ abs_path)

        val os = TextIO.openOut abs_path;
        val _ = TextIO.output (os, str);
        val _ = TextIO.flushOut os;
        val _ = TextIO.closeOut os;
        in thy end
      end
  | export_c_code NONE _ _ thy =
      (error "Invalid program, no code is generated."; thy)
\end{lstlisting}

Esta función toma cuatro parámetros.
El primer parámetro de la función \verb|export_c_code| es un \verb|string option| esto corresponde a la representación en cadena de caracteres del código C generado por el programa.
Si recibe un valor \verb|None| significa que la función \verb|prepare_export| falló y no se debe generar un programa en C.
El segundo parámetro es la ruta al directorio al cual se quiere exportar el programa, este parámetro es una ruta relativa al directorio donde se encuentra la teoría de Isabelle que contiene las directivas para hacer \textit{pretty printing}.
El tercer parámetro es el nombre del programa.
El último parámetro es el contexto de la teoría actual.

Si la ruta dada es una cadena vacía el código generado se imprimirá a la vista de salida de Isabelle.
Esta función crea un nuevo archivo con el nombre indicado en los parámetros con ``\verb|.c|'' agregado al final (es decir, $<nombre>\mathtt{.c}$) en el directorio indicado por el parámetro de la ruta.
El usuario puede entonces encontrar el código generado en el directorio indicado.
\end{comment}

También se define una función en ML que se utiliza para el caso donde se escribe un programa erróneo a propósito y se espera que su ejecución y traducción a C fallen.
Esta función genera un error si se genera código a partir de un programa para el cual se esperaba una falla y no realizará acción alguna cuando no se genere código.

\begin{comment}
\begin{lstlisting}
fun expect_failed_export (SOME _) = error ``Expected failed export''
  | expect_failed_export NONE = ()
\end{lstlisting}

La función \verb|expect_failed_export| genera un error en Isabelle si se genera código a partir del programa para el cual se esperaba una falla y no realizará acción alguna cuando no se genere código.
Tener este tipo de función será útil más adelante para la realización de pruebas.
\end{comment}
