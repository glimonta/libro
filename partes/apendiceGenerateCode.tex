\chapter{Generación de código en C}
\label{ap:generate_c_code}
\lhead{Apéndice F. \emph{Generación de código en C}}
\begin{figure}
\begin{lstlisting}
  fun export_c_code (SOME code) rel_path name thy =
    let
      val str = code |> String.implode;
    in
      if rel_path="" orelse name="" then
        (writeln str; thy)
      else let
        val base_path = Resources.master_directory thy
        val rel_path = Path.explode rel_path
        val name_path = Path.basic name |> Path.ext "c"

        val abs_path = Path.appends [base_path, rel_path, name_path]
        val abs_path = Path.implode abs_path

        val _ = writeln ("Writing to file " ^ abs_path)

        val os = TextIO.openOut abs_path;
        val _ = TextIO.output (os, str);
        val _ = TextIO.flushOut os;
        val _ = TextIO.closeOut os;
        in thy end
      end
  | export_c_code NONE _ _ thy =
      (error "Invalid program, no code is generated."; thy)
\end{lstlisting}
\caption{Generación de código en C desde Isabelle/HOL}
\end{figure}

Esta función toma cuatro parámetros.
El primer parámetro de la función \verb|export_c_code| es un \verb|string option| esto corresponde a la representación en cadena de caracteres del código C generado por el programa.
Si recibe un valor \verb|None| significa que la función \verb|prepare_export| falló y no se debe generar un programa en C.
El segundo parámetro es la ruta al directorio al cual se quiere exportar el programa, este parámetro es una ruta relativa al directorio donde se encuentra la teoría de Isabelle que contiene las directivas para hacer \textit{pretty printing}.
El tercer parámetro es el nombre del programa.
El último parámetro es el contexto de la teoría actual.

Si la ruta dada es una cadena vacía el código generado se imprimirá a la vista de salida de Isabelle.
Esta función crea un nuevo archivo con el nombre indicado en los parámetros con ``\verb|.c|'' agregado al final (es decir, $<nombre>\mathtt{.c}$) en el directorio indicado por el parámetro de la ruta.
El usuario puede entonces encontrar el código generado en el directorio indicado.
