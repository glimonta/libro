\chapter{Conclusiones y trabajo futuro}\label{chapter:conclusion}
\lhead{Capítulo 6. \emph{Conclusiones y trabajo futuro}}

\section{Conclusiones}

En este trabajo se logró formalizar exitosamente la semántica para un lenguaje imperativo llamado Chloe, el cual abarca un subconjunto del lenguaje C.
Chloe tiene las siguientes características: variables, arreglos, aritmética de apuntadores, ciclos, condicionales, funciones y memoria dinámica.
Se formalizó una semántica de pasos cortos en el probador de teoremas Isabelle/HOL para Chloe y se demostró que la misma es determinística.
Además, se presenta un interpretador para el lenguaje, lo que permite que los programas escritos en Chloe sean ejecutados en el entorno de Isabelle/HOL.
Para escribir el interpretador era necesario demostrar que la semántica era determinística.

Otro resultado de este trabajo fue la presentación de un generador de código para el lenguaje Chloe.
Este generador de código traduce programas de la semántica formal a código C real que puede ser ejecutado.
Se garantiza que el programa generado puede ser compilado y ejecutado en una máquina si no posee comportamiento indefinido, cumple con las restricciones supuestas en la formalización de la semántica (sección~\ref{subsection:restrictions_commands}) y no se queda sin memoria durante la ejecución.
Además, solo se genera código en C para programas que sean válidos y se ejecuten correctamente en el interpretador.

Al formalizar la semántica se presentaron varios problemas, tales como sólo permitir que programas con comportamiento definido fueran ejecutados en la semántica.
Esto significa que se tuvo que detectar comportamiento indefinido, de acuerdo al estándar de C~\citep{c99}, tales como el \textit{overflow} de enteros y considerarlo como un error en la semántica.
Otro problema se presentó al formalizar la semántica para la función de asignación de memoria ya que se supone que una cantidad ilimitada de memoria está disponible, mientras que los recursos en una máquina son limitados.
Para solventar esta problemática, se decidió cambiar la manera en que se traducía una llamada a la función de asignación de memoria a código C.
Se rodea la función \verb|malloc| de C en una nueva definida por el usuario que aborta el programa si una llamada a \verb|malloc| falla.
También se supone una restricción sobre la arquitectura de la máquina destino donde el código será ejecutado y se genera código que verifica que la máquina satisface las suposiciones.


Por otra parte, se presenta un arnés de pruebas y una \textit{suite} de pruebas para el proceso de traducción.
La finalidad de esta \textit{suite} de pruebas es incrementar la confianza en el proceso de traducción y asegurarse que la semántica de un programa en Chloe no cambia al ser traducido a código C.
En la \textit{suite} de pruebas se incluyen casos incorrectos que cubren los casos bordes o casos donde no se debería generar código C ya que el programa encuentra un error.
Además, se incluyen programas de ejemplo para demostrar como funciona el lenguaje y como se traducen los programas a código C.
Existe también un conjunto de pruebas correctas para las que se espera generar código.
La batería completa de pruebas puede ser encontrada en el código fuente presentado con este trabajo.

Para garantizar que la semántica de un programa no es cambiada por el proceso de traducción, se incluye una manera de generar código con chequeos.
Estos chequeos tienen la finalidad de verificar que el estado final de un programa ejecutado en el interpretador dentro de Isabelle/HOL es equivalente al estado final del programa ejecutado y compilado.
Dos estados son considerados equivalentes cuando las variables globales y la memoria dinámica accesible, al final de la ejecución, tienen el mismo contenido.
Para ello, se escribe un arnés de pruebas en Isabelle que traduce las pruebas que se deben hacer a un conjunto de macros en C que se definen fuera de Isabelle/HOL, y se incluyen en el proceso de compilación, que verifican la equivalencia entre estados finales.
Con el fin de no correr estas pruebas manualmente se crea un \textit{bash script} que corre todas las pruebas existentes automáticamente.

\section{Trabajo futuro}

En esta sección se señalan algunas direcciones que puede tomar este trabajo en un futuro.
El marco de tiempo disponible para hacer este trabajo hizo imposible incluir estas características en el alcance del mismo.

En primer lugar, se puede mejorar el proceso de pruebas al reconocer lexicográficamente pruebas de C de una \textit{suite} de pruebas externa y traducirlas a Chloe.
Esto permitiría ofrecer métricas mas exactas para los resultados del proceso de pruebas, por ejemplo, que tanto de la \textit{suite} de pruebas abarca exitosamente el trabajo.
Un ejemplo de esto sería tomar una \textit{suite} de pruebas hecha para compiladores de C, tales como las \textit{suites} de prueba de gcc~\citep{gcc-tests} y reducir las pruebas a aquellas que puedan ser traducidas a la semántica, traducirlas de C a Chloe y generar código con chequeos a partir de las mismas.
Luego de traducir a Chloe se puede generar código con chequeos para las mismas con el fin de mejorar la \textit{suite} de pruebas que se presenta con este trabajo.

Otra dirección interesante que se podría tomar es la de formalizar una semántica axiomática con el fin de razonar sobre programas y sus propiedades en Chloe.
Dado que Chloe posee apuntadores como una de sus características, sería necesario formalizar una lógica de separación~\citep{sep-logic} para poder razonar sobre apuntadores en los programas.
Al extender el trabajo en esta dirección sería posible demostrar propiedades de correctitud parcial y total para los programas.

Una de las características que no se encuentran incluidas en el alcance del trabajo es un sistema de tipos estático que sea demostrado como correcto y sólido.
Esto permitiría razonar sobre seguridad de tipos para los programas escritos en Chloe.

Existe un marco de trabajo llamado Isabelle Refinement Framework~\citep{Refine_Monadic-AFP}, que proporciona una manera de formular algoritmos no determinísticos en un estilo monádico y refinarlos para obtener un algoritmo ejecutable.
Proporciona herramientas para razonar sobre estos programas.
Otra fuente de trabajo futuro sería vincular este lenguaje al Isabelle Refinement Framework, de modo que los programas del marco de trabajo puedan ser refinados a programas en Chloe.

Por último, el conjunto de características que actualmente son compatibles con Chloe es limitado.
Otra forma de mejorar los resultados de este trabajo es aumentar el conjunto de características compatibles con Chloe.
Esto puede incluir expandir el conjunto de expresiones e instrucciones (por ejemplo, agregar compatibilidad para structs y unions), agregar operaciones de entrada y salida o agregar soporte para concurrencia.
