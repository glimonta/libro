\chapter{\textit{Pretty Printing}}
\label{ap:pretty_printing}
\lhead{Apéndice E. \emph{\textit{Pretty Printing}}}

\section{Palabras}\label{section:pretty_words}
La primera instanciación de \verb|shows| que se debe hacer es aquella para las palabras de modo que sea posible imprimir valores de este tipo.
Para hacer \textit{pretty printing} de un valor del tipo palabra simplemente se hace \textit{casting} de ese valor a un tipo entero predefinido por Isabelle/HOL y se utiliza la función \verb|show| para ese tipo.
Como resultado, las palabras serán impresas como enteros con signo.

\section{Valores}\label{section:pretty_values}

Aunque el tipo de datos \verb|val| y las valuaciones no son utilizadas en el proceso de generación de código, se considera útil tener un mecanismo para hacer \textit{pretty printing} de los mismos con fines de depuración.

\subsection{Tipo val}\label{subsection:pretty_val_type}
El tipo \verb|val| es el primer tipo de datos definido por el usuario para el que se presenta una instanciación.
En la tabla~\ref{tab:pretty_val} se encuentra la equivalencia entre la sintaxis abstracta para el tipo \verb|val| y la representación en cadena de caracteres.
Nótese que $w$, $base$ y $ofs$ representan palabras, naturales y enteros, por lo tanto sus funciones \verb|show| serán utilizadas para obtener su representación en cadena de caracteres, por ejemplo, la representación en cadena de caracteres de $\mathtt{I}\ 42$ sería $"42"$, la representación en cadena de caracteres de $\mathtt{A}\ (4,\ 2)$ sería $"4[2]"$.
Se puede realizar \verb|show| de una lista de valores al hacer \verb|show| de cada valor en una notación de listas, es decir, $[vname\ =\ valor, \dots, vname_n\ =\ valor_n]$.

Se rodean con cuñas los parámetros ($"<\ >"$) para indicar que lo que está dentro de ellos es una representación en forma de cadena de caracteres resultante de aplicar una función \verb|show| en el parámetro y se continuará usando esta notación a lo largo del documento.

\begin{table}[h!]
\centering
\begin{tabular}{|c|c|}
  \hline
  \textbf{Sintaxis abstracta} & \textbf{Representación en cadena de caracteres} \\ [0.5ex]
  \hline \hline
  \verb|NullVal| & null \\
  \verb|I| $w$ & $<w>$ \\
  \verb|A| $(base,\ ofs)$ & $<base>$[$<ofs>$] \\
  \hline
\end{tabular}

\caption{Traducción del tipo val}
\label{tab:pretty_val}
\end{table}

\subsection{Tipo val option}\label{subsection:pretty_val_option_type}

Un \verb|val option| tiene el significado semántico de un valor inicializado y no inicializado.
La tabla~\ref{tab:pretty_val_option} muestra la equivalencia entre la sintaxis abstracta y la representación en cadena de caracteres.
Un valor inicializado simplemente será la cadena de caracteres resultante de llamar a la función \verb|show| sobre ese valor, mientras que la representación de un valor no inicializado será un $''?''$ para representar que el valor es aun desconocido.

\begin{table}[h!]
\centering
\begin{tabular}{|c|c|}
  \hline
  \textbf{Sintaxis abstracta} & \textbf{Representación en cadena de caracteres} \\ [0.5ex]
  \hline \hline
  \verb|Some| v & $<v>$ \\
  \verb|None| & ? \\
  \hline
\end{tabular}

\caption{Traducción del tipo val option}
\label{tab:pretty_val_option}
\end{table}


\subsection{Valuaciones}\label{subsection:pretty_valuations}

Para representar una valuación se necesita un parámetro extra: una lista de nombres de variables, esta lista tendrá los nombres de las variables para los cuales se quiere imprimir su valor en la valuación.
La valuación será impresa con el siguiente formato:

\begin{equation*}
[<vname_0> = <valor_0>, <vname_1> = <valor_1>, \dots, <vname_n> = <valor_n>]
\end{equation*}

Por ejemplo, si se toma la valuación $[foo \mapsto \mathtt{Some}(\mathtt{I}\ 15), bar \mapsto None, baz \mapsto \mathtt{Some}(\mathtt{A}(1,9))]$ y la lista de variables $[foo, bar, baz]$ entonces se obtiene la siguiente representación para la valuación: $[foo\ =\ 15,\ bar\ =\ ?,\ baz\ =\ 1[9]]$.

\section{Memoria}\label{section:pretty_memory}

La memoria, al igual que los valores, no es un componente necesario para generar programas en C.
No obstante, tener una manera de hacer \textit{pretty printing} de la memoria en cierto estado es útil al de realizar depuración.
Para imprimir la memoria se debe saber como hacer \verb|show| del contenido de un bloque y de todo el bloque.

Cuando se accede a un bloque a memoria se puede obtener tanto el contenido del bloque o un valor \verb|None| que indica un bloque libre en memoria.
La tabla~\ref{tab:pretty_block_content} muestra la representación en cadena de caracteres de esto.
Nótese que en el caso de un valor \verb|None| se imprime la cadena \verb|free| rodeada de cuñas, siendo las mismas parte de la representación y una excepción a la notación descrita anteriormente.
Si el contenido del bloque es una lista de valores entonces se imprime la lista de valores.

\begin{table}[h!]
\centering
\begin{tabular}{|c|c|}
  \hline
  \textbf{Sintaxis abstracta} & \textbf{Representación en cadena de caracteres} \\ [0.5ex]
  \hline \hline
  \verb|Some| content & $<content>$ \\
  \verb|None| & $<$free$>$ \\
  \hline
\end{tabular}

\caption{Traducción del contenido de un bloque}
\label{tab:pretty_block_content}
\end{table}

Un bloque completo se imprime de la siguiente manera:
\begin{equation*}
<base>\ :\ <contenido\_bloque> 
\end{equation*}

donde $base$ es el primer componente de una dirección con la que se accede al bloque y $contenido\_bloque$ es la representación en cadena de caracteres del contenido en el bloque número $base$.

Para imprimir la memoria completa se hace \verb|show| de cada bloque existente en memoria.
Por ejemplo, la representación en cadena de caracteres del estado de memoria: $[\mathtt{Some}\ [\mathtt{I}\ 13,\ \mathtt{None}],\ None,\ \mathtt{Some}\ [\mathtt{A}(2,3),\ None, \mathtt{I}\ 56]]$ sería:

\begin{align*}
0\ &:\ [13,\ ?] \\
1\ &:\ <free> \\
2\ &:\ [2[3],\ ?,\ 56]
\end{align*}

\section{Expresiones}

\paragraph*{Operaciones unarias y binarias}
Al hacer \textit{pretty printing} de operaciones binarias se utilizan paréntesis alrededor de cada expresión que se imprime.
Esto, naturalmente, generará más paréntesis de los necesarios pero se toma esta decisión para asegurar que el orden de evaluación se mantenga igual al previsto y que no se obtengan diferentes órdenes de evaluación debido a la precedencia de los operadores.

Un operador binario se imprime usando notación de infijo: $(<operando_1> <operador> <operando_2>)$.
Ejemplos de esto se muestran en la tabla~\ref{tab:pretty_bin_op}.

\begin{table}[h!]
\centering
\begin{tabular}{|c|c|}
  \hline
  \textbf{Sintaxis abstracta} & \textbf{Representación en cadena de caracteres} \\ [0.5ex]
  \hline \hline
  \verb|Plus| (\verb|Const| $11$) (\verb|Const| $11$) & ($11\ +\ 11$) \\
  \verb|Subst| (\verb|Const| $9$) (\verb|Const| $5$) & ($9\ -\ 5$) \\
  \verb|Mult| (\verb|Const| $2$) (\verb|Const| $3$) & ($2\ *\ 3$) \\
  \hline
\end{tabular}

\caption{Ejemplos de \textit{pretty printing} de operadores binarios}
\label{tab:pretty_bin_op}
\end{table}

Un operador unario se imprime utilizando notación de prefijo: $<operador> (<operando>)$.
Nótese que se rodea entre paréntesis el operando con el fin de garantizar la correcta precedencia en las operaciones.
Ejemplos de esto se muestran en la tabla~\ref{tab:pretty_un_op}.

\begin{table}[h!]
\centering
\begin{tabular}{|c|c|}
  \hline
  \textbf{Sintaxis abstracta} & \textbf{Representación en cadena de caracteres} \\ [0.5ex]
  \hline \hline
  \verb|Minus| (\verb|Const| $11$) & $-\ (11)$ \\
  \verb|Not| (\verb|Const| $0$) & $!\ (0)$ \\
  \hline
\end{tabular}

\caption{Ejemplos de \textit{pretty printing} de operadores unarios}
\label{tab:pretty_un_op}
\end{table}


\paragraph*{Expresiones}
En la tabla~\ref{tab:pretty_expressions} se presenta la representación en cadena de caracteres para cada una de las expresiones.
Se utilizan operaciones simples como operandos tales como variables o valores constantes por simplicidad, sin embargo las expresiones pueden estar compuestas por términos más complejos.

\begin{table}[h!]
\centering
\begin{tabular}{|c|c|}
  \hline
  \textbf{Sintaxis abstracta} & \textbf{Representación en cadena de caracteres} \\ [0.5ex]
  \hline \hline
  \verb|Const| $42$ & $42$ \\
  \verb|Null| & (\verb|intptr_t| *) $0$ \\
  \verb|V| $x$ & $x$ \\
  \verb|Plus| $(\mathtt{Const}\ 2)\ (\mathtt{Const}5)$ & $(2\ +\ 5)$ \\
  \verb|Subst| $(\mathtt{Const}\ 9\ (\mathtt{Const}5)$ & $(9\ -\ 5)$ \\
  \verb|Minus| $(\mathtt{Const}\ 9)$ & ($-9$) \\
  \verb|Div| $(\mathtt{Const}\ 8)\ (\mathtt{Const}\ 4)$ & $(8\ /\ 4)$ \\
  \verb|Mod| $(\mathtt{Const}\ 8)\ (\mathtt{Const}\ 4)$ & $(8\ \%\ 4)$ \\
  \verb|Mult| $(\mathtt{Const}\ 9)\ (\mathtt{Const}\ 3)$ & $(9\ *\ 3)$ \\
  \verb|Less| $(\mathtt{Const}\ 7)\ (\mathtt{Const}\ 9)$ & $(7\ <\ 9)$ \\
  \verb|Not| $(\mathtt{Const}\ 0)$ & $!\ (0)$ \\
  \verb|And| $(\mathtt{Const}\ 1)\ (\mathtt{Const}\ 1)$ & $(1\ \&\&\ 1)$ \\
  \verb|Or| $(\mathtt{Const}\ 1)\ (\mathtt{Const}\ 0)$ & $(1\ || 0)$ \\
  \verb|Eq| $(\mathtt{Const}\ 6)\ (\mathtt{Const}\ 4)$ & $(6\ == 4)$ \\
  \verb|New| $(\mathtt{Const}\ 9)$ & $(\mathtt{intptr\_t})\ \mathtt{\_\_MALLOC}(\mathtt{sizeof}(\mathtt{intptr\_t)}\ *\ ($9$))$ \\
  \verb|Deref| $(\mathtt{V}\ foo)$ & $*((\mathtt{intptr\_t}\ *)\ foo)$ \\
  \verb|Ref| $(\mathtt{V}\ foo)$ & $((\mathtt{intptr\_t}\ *)\ \&(foo))$ \\
  \verb|Index| $(\mathtt{V}\ bar)\ (\mathtt{Const}\ 3)$ & $((\mathtt{intptr\_t}\ *)\ bar[3])$ \\
  \verb|Derefl| $(\mathtt{V}\ foo)$ & $(*((\mathtt{intptr\_t}\ *)\ foo))$ \\
  \verb|Indexl| $(\mathtt{V}\ bar)\ (\mathtt{Const}\ 3)$ & $((\mathtt{intptr\_t}\ *)\ bar[3])$ \\
  \hline
\end{tabular}

\caption{Ejemplos de \textit{pretty printing} para expresiones}
\label{tab:pretty_expressions}
\end{table}


\section{Comandos}

Primeramente, se necesita un mecanismo que permita imprimir comandos indentados para facilitar la generación de código.
Se definen dos abreviaciones auxiliares que imprimen espacios en blanco para indentación al inicio de un término.
La razón por la que se definen dos abreviaciones es porque una de ellas también imprime un terminador $";"$ luego del término, mientras que la otra no.

Las llamadas a funciones se imprimen siguiendo el siguiente formato: $<nombre\_funcion>\ ([<argumento_0, argumento_1, \dots, <argumento_n>])$ donde los corchetes ($[]$) indican que los argumentos son opcionales.

Los comandos en Chlose se imprimen como se muestra en los ejemplos de la tabla~\ref{tab:pretty_commands}.
Se utiliza $''$ para indicar una cadena vacía.
El nivel correcto de indentación se indica en los parámetros de la función que se encarga de realizar el \textit{pretty printing}, acá esos son omitidos y en su lugar se indica donde la indentación aumenta.
La indentación aumenta al imprimir comandos que se encuentren dentro de un bloque, por ejemplo: las ramas de un condicional o el cuerpo de un ciclo.

\begin{table}[h!]
\centering
\begin{tabular}{|c|l|}
  \hline
  \textbf{Sintaxis abstracta} & \textbf{Representación en cadena de caracteres} \\ [0.5ex]
  \hline \hline
  \verb|SKIP| & $''$ \\
  \hline
  $\mathtt{Derefl}\ foo\ ::==\ \mathtt{Const}\ 4$                            & $*((\mathtt{intptr\_t}\ *)\ foo)\ =\ 4;$ \\
  \hline
  $foo\ ::=\ \mathtt{Const}\ 4$                                              & $foo\ =\ 4;$ \\
  \hline
  $c_1;;\ c_2$                                                               & $<c_1> <c_2>$ \\
  \hline
  $\mathtt{IF}\ (\mathtt{V}\ b)\ \mathtt{THEN}$                              & $\mathtt{if}\ (b)\ \{$\\
  $foo\ ::=\ \mathtt{Const}\ 4$                                              & $\ \ foo\ =\ 4;$ \\
  $\mathtt{ELSE}$                                                            &  $\}$ \\
  $\mathtt{SKIP}$                                                            & \\
  \hline
  $\mathtt{If}\ (\mathtt{V}\ b)\ \mathtt{THEN}$                              & $\mathtt{if}\ (b)\ \{$\\
  $foo\ ::=\ \mathtt{Const}\ 4)$                                             & $\ \ foo\ =\ 4;$ \\
  $\mathtt{ELSE}$                                                            & $\}\ else\ \{$ \\
  $bar\ ::=\ \mathtt{Const}\ 3)$                                             & $\ \ bar\ =\ 3;$ \\
                                                                             & $\}$ \\
  \hline
  $\mathtt{While}\ (\mathtt{V}\ b)\ \mathtt{DO}$                             & $\mathtt{while}\ (b)\ \{$\\
  $foo\ ::=\ \mathtt{Const}\ 4$                                              & $\ \ foo\ =\ 4;$ \\
                                                                             & $\}$ \\
  \hline
  \verb|FREE| $(\mathtt{Derefl}\ foo)$                                       & $\mathtt{free}\ (\&\ (*((\mathtt{intptr\_t}\ *)\ foo));$ \\
  \hline
  \verb|RETURN| $(\mathtt{V}\ foo)$                                          & $\mathtt{return}\ (foo));$ \\
  \hline
  \verb|RETURNV|                                                             & $\mathtt{return};$ \\
  \hline
  $(\mathtt{Derefl}\ foo)\ ::==\ bar\ ([\mathtt{V}\ baz,\mathtt{Const}\ 4])$ & $*((\mathtt{intptr\_t}\ *)\ foo)\ =\ bar(baz,\ 4);$ \\
  \hline
  $foo\ ::=\ bar\ ([\mathtt{Const}\ 65])$                                    & $foo =\ bar(65);$ \\
  \hline
  \verb|CALL| $bar\ ([])$                                                    & $bar();$ \\
  \hline
\end{tabular}

\caption{Ejemplos de \textit{pretty printing} para comandos}
\label{tab:pretty_commands}
\end{table}


\section{Declaraciones de funciones}

A continuación se define como se imprimen las declaraciones de funciones.
Para hacerlo, se imprime la definición de una función de acuerdo al siguiente formato:
\begin{equation*}
\begin{split}
&\mathtt{intptr\_t} \ <nombre\_de\_funcion>(\mathtt{intptr\_t}\ <nombre\_arg_0>,\ \dots,\\
&\ \ \ \ \ \ \ \ \ \ \ \ \ \ \ \ \ \ \ \ \ \ \mathtt{intptr\_t}\ <nombre\_arg_n)\ \{ \\
&\ \ \mathtt{intptr\_t}\ <var\_local_0> \\
&\vdots \\
&\ \ \mathtt{intptr\_t}\ <var\_local_n> \\
&<cuerpo> \\
&\} \\
\end{split}
\end{equation*}

El retorno, los argumentos y las variables locales es de tipo \verb|intptr_t| dado que, como se mencionó anteriormente, solo se tiene un tipo en el proceso de traducción y se realizan \textit{casts} hacia y desde apuntadores cuando son necesarios.

Un ejemplo de la traducción de la declaración de una función para una función factorial se encuentra en la tabla~\ref{tab:pretty_function_fact}.
En este ejemplo se evita el uso de $''\ ''$ para representar cadenas de caracteres en Isabelle/HOL y simplemente se escribe la cadena de caracteres sin las comillas.

\begin{table}[h!]
\centering
\begin{tabular}{|l|l|}
  \hline
  \textbf{Sintaxis abstracta} & \textbf{Representación en cadena de caracteres} \\ [0.5ex]
  \hline \hline
  \verb|definition factorial_decl| $::$ \verb|fun_decl|  & \verb|intptr_t fact(intptr_t n) {| \\
  \verb|where "factorial_decl| $\equiv$                  & \verb|  intptr_t r;| \\
  \verb|  ( fun_decl.name = fact,|                       & \verb|  intptr_t i;| \\
  \verb|    fun_decl.params = [n],|                      & \verb|  r = (1);| \\
  \verb|    fun_decl.locals = [r, i],|                   & \verb|  i = (1);| \\
  \verb|    fun_decl.body =|                             & \verb|  while ((i) < ((n) + (1))) {| \\
  \verb|      r ::= (Const 1);;|                         & \verb|    r = ((r) * (i));| \\
  \verb|      i ::= (Const 1);;|                         & \verb|    i = ((i) + (1));| \\
  \verb|      (WHILE|                                    & \verb|  }| \\
  \verb|         (Less (V i) (Plus (V n) (Const 1)))|    & \verb|  return(r);| \\
  \verb|      DO|                                        & \verb|}| \\
  \verb|        (r ::= (Mult (V r) (V i));;|             & \\
  \verb|        i ::= (Plus (V i) (Const 1)))|           & \\
  \verb|      );;|                                       & \\
  \verb|    RETURN (V r)|                                & \\
  \verb|  )"|                                            & \\
  \hline
\end{tabular}

\caption{Ejemplo de \textit{pretty printing} de la declaración de una función factorial}
\label{tab:pretty_function_fact}
\end{table}


\section{Estados}

Al ejecutar un programa dentro del ambiente de Isabelle/HOL a menudo se quieren inspeccionar los estados.
En esta sección se define una forma sencilla de inspeccionar los estados al tener una representación en cadena de caracteres para los mismos.

Primero se describe como se imprime una ubicación de retorno.
Se instancia la clase \verb|show| para el tipo \verb|return_loc|.
En la tabla~\ref{tab:pretty_rloc} se encuentra la equivalencia entre la sintaxis abstracta para el tipo \verb|return_loc| y su representación en cadena de caracteres.
Donde $<base>$ y $<ofs>$ son el resultado de aplicar la función \verb|show| sobre $base$ y $ofs$ y $<invalid>$ es una cadena de caracteres literal que incluye las cuñas.

\begin{table}[h!]
\centering
\begin{tabular}{|c|c|}
  \hline
  \textbf{Sintaxis abstracta} & \textbf{Representación en cadena de caracteres} \\ [0.5ex]
  \hline \hline
  \verb|Ar| $(base,\ ofs)$ & $<base>$[$<ofs>$] \\
  \verb|Vr| $w$ & $w$ \\
  \verb|Invalid| & $<$invalid$>$ \\
  \hline
\end{tabular}

\caption{Traducción del tipo de una ubicación de retorno}
\label{tab:pretty_rloc}
\end{table}

Luego se describe como se imprime la pila.
Un solo marco de pila se imprime al imprimir el comando, la lista de variables locales y la ubicación de retorno esperada con el siguiente formato: $rloc\ =\ <rloc>$ separados por un salto de línea.
Para imprimir la pila completa, se imprime cada marco de pila separado por ``\verb|---------------|''.


Para imprimir un estado se debe dar a la función \verb|show| una lista de nombres de variables, la cual será utilizada para imprimir la valuación para las variables globales y para las locales en cada marco de pila.
Un estado se imprime al imprimir la pila, los valores de las variables globales y, finalmente, la memoria, separados por ``\verb|=================|''
Un ejemplo de \textit{pretty printing} para un estado se muestra en la tabla~\ref{tab:pretty_simp_state}.

\begin{table}[h!]
\centering
\begin{tabular}{|l|l|}
  \hline
  \textbf{Sintaxis abstracta} & \textbf{Representación en cadena de caracteres} \\ [0.5ex]
  \hline \hline
  $(\ [\ (x\ ::=\ \mathtt{Const}\ 4;;\ y\ ::=\ \mathtt{Const}\ 25,$                                            & \verb|x = (4);| \\
  $\ \ \ \ \ \ [z\ \mapsto\ \mathtt{Some}\ (\mathtt{I}\ 0)],\ \mathtt{Invalid}),$                              & \verb|y = (25);| \\
  $\ \ [\ (x\ ::=\ \mathtt{Const}\ 3;;\ y\ ::=\ \mathtt{Const}\ 43,$                                           & \verb|---------------| \\
  $\ \ \ \ \ \ [z\ \mapsto\ \mathtt{Some}\ (\mathtt{I}\ 6)],\ \mathtt{Vr} foo)],$                              & \verb|[z = 0]| \\
  $\ \ [x\ \mapsto\ \mathtt{Some}\ (\mathtt{I}\ 3),\ y\ \mapsto\ \mathtt{Some}\ (\mathtt{I}\ 8),$              & \verb|---------------| \\
  $\ \ foo\ \mapsto\ \mathtt{Some}\ (\mathtt{I}\ 0)],$                                                         & \verb|rloc = <invalid>| \\
  $\ \ [\mathtt{None},\ \mathtt{Some}\ [\mathtt{Some}\ (\mathtt{I}\ 44), \mathtt{Some}\ (\mathtt{A}\ (2,0))],$ & \verb|x = (5);| \\
  $\ \ \mathtt{Some}\ [\mathtt{Some}\ (\mathtt{I}\ 78)]$                                                       & \verb|y = (43);| \\
                                                                                                               & \verb|---------------| \\
                                                                                                               & \verb|[z = 6]| \\
                                                                                                               & \verb|---------------| \\
                                                                                                               & \verb|rloc = foo| \\
                                                                                                               & \verb|=================| \\
                                                                                                               & \verb|[x = 3, y = 8, foo = 0]| \\
                                                                                                               & \verb|=================| \\
                                                                                                               & \verb|1: <free>| \\
                                                                                                               & \verb|2: [44, *2[0]]| \\
                                                                                                               & \verb|3: [78]| \\
  \hline
\end{tabular}

\caption{Ejemplo de \textit{pretty printing} para un estado}
\label{tab:pretty_simp_state}
\end{table}

