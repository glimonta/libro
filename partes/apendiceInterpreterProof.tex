\chapter{Demostraciones de lemas para el interpretador}
\label{ap:interpreter}
\lhead{Apéndice D. \emph{Demostraciones de lemas para el interpretador}}


Se comienza demostrando que cualquier paso tomado en la semántica de pasos cortos puede ser simulado a través de un paso tomado con \verb|fstep|.

\begin{lemma}[fstep1]
$\newline$
$s\ \rightarrow\ s'\ \Longrightarrow\ \mathtt{fstep}\ s\ =\ s'$
\label{lemma:fstep1}
\end{lemma}

\begin{proof}
La demostración es por inducción sobre la semántica de pasos cortos.
\end{proof}

Luego se considera la dirección opuesta:

\begin{lemma}[fstep2]
$\newline$
$\neg\ \mathtt{is\_empty\_stack}\ s\ \Longrightarrow\ s\ \rightarrow\ (\mathtt{fstep}\ s)$
\label{lemma:fstep2}
\end{lemma}

\begin{proof}
La demostración se hace automáticamente mediante una prueba por casos sobre el resultado de ``\verb|tr_return_void| $s$'' y utilizando los lemas~\ref{lemma:can_take_step} y~\ref{lemma:fstep1}.
\end{proof}

Ambas direcciones juntas (lema~\ref{lemma:fstep1} y lema~\ref{lemma:fstep2}) permiten demostrar la equivalencia planteada al inicio:

\begin{lemma}[ss\_fstep\_equiv]
$\newline$
$\neg\ \mathtt{is\_empty\_stack}\ \Longrightarrow\ s\ \rightarrow\ s'\ \longleftrightarrow\ \mathtt{fstep}\ s\ =\ s'$
\label{lemma:ss_fstep_equiv}
\end{lemma}

Se presenta un lema que establece que si un estado es final, entonces es el resultado de la interpretación.

\begin{lemma}[interp\_term]
$\newline$
$\mathtt{is\_term}\ \mathtt{(Some}\ cs\mathtt{)}\ \Longrightarrow\ \mathtt{interp}\ \mathtt{proc\_table}\ cs\ =\ \mathtt{Some}\ cs$
\label{lemma:interp_term}
\end{lemma}

Para poder demostrar esto necesitamos un lema que expanda la definición del ciclo en la definición de nuestro interpretador:

\begin{lemma}[interp\_unfold]
$\newline$
$\mathtt{interp}\ \mathtt{proc\_table}\ cs = ($
$\newline$
$\mathtt{if}\ \mathtt{is\_term}\ (\mathtt{Some}\ cs)\ \mathtt{then}\ \mathtt{Some}\ cs$
$\newline$
$\mathtt{else}\ \mathtt{do\{}\ cs \leftarrow\ \mathtt{fstep}\ \mathtt{proc\_table}\ cs\mathtt{;}\ \mathtt{interp}\ \mathtt{proc\_table}\ cs$
$\mathtt{\}})$
\label{lemma:interp_unfold}
\end{lemma}

\begin{proof}
La demostración es hecha automáticamente.
\end{proof}

Con el lema~\ref{lemma:interp_unfold}, la demostración del lema~\ref{lemma:interp_term} se hace automáticamente.

Antes de demostrar el lema de correctitud para el interpretador, se debe demostrar que la semántica de pasos cortos preserva un estado erróneo \verb|None| si tal es alcanzado por el camino de pasos tomados.
Tras una ejecución errónea, la misma se queda atascada en un estado \verb|None|.

\begin{lemma}[None\_star\_preserved]
$\newline$
$\mathtt{None}\ \rightarrow*\ z\ \longleftrightarrow\ z\ =\ \mathtt{None}$
\label{lemma:none_star_preserved}
\end{lemma}

\begin{proof}
La demostración es por inducción sobre la clausura reflexivo transitiva (\verb|star|).
Los objetivos se resuelven automáticamente.
\end{proof}

Finalmente se tiene la propiedad de correctitud para el interpretador.
El teorema~\ref{theorem:interp_correct} establece que si la ejecución de $cs$ termina, entonces esa ejecución produce $cs'$ si y solo si $cs'$ es el resultado que se obtiene de ejecutar el programa en el interpretador.

\begin{theorem}[interp\_correct]
$\newline$
$\mathtt{terminates}\ cs\ \Longrightarrow\ (\mathtt{yields}\ cs\ cs')\ \longleftrightarrow\ (cs'\ =\ \mathtt{interp}\ \mathtt{proc\_table}\ cs)$
\label{theorem:interp_correct}
\end{theorem}

\begin{proof}
La demostración se hace suponiendo el antecedente y demostrando cada dirección de la igualdad por separado.
\end{proof}
