\chapter{Introducción}\label{chapter:introduction}
\lhead{Capítulo 1. \emph{Introducción}}

En este capítulo se discute la motivación para realizar este trabajo.
Seguido de una breve descripción de los conceptos relacionados a semántica e Isabelle/HOL.

Luego, se describen las características del lenguaje utilizado en este trabajo.

Finalmente, se describe el contenido del resto del trabajo.

\section{Motivación}

El objetivo de este trabajo es formalizar al semántica de un lenguaje imperativo (incluyendo apuntadores y arreglos) que representa un subconjunto del lenguaje C y luego, generar código ejecutable del mismo.

C es un lenguaje ampliamente utilizado.
Es especialmente popular en la implementación de sistemas operativos y aplicaciones de sistemas embebidos.
Dado que C es mas cercano a la máquina en comparación con otros lenguajes de alto nivel y presenta bajo \textit{overhead}, permite la implementación eficiente de algoritmos y estructuras de datos.
Debido a su eficiencia, a menudo es usado en el desarrollo de compiladores, librerías e interpretadores para otros lenguajes de programación.

Desafortunadamente, parte de la semántica para el lenguaje C descrita en el estándar~\cite{c99} es propensa a ambigüedades debido al uso del idioma inglés para describir el comportamiento de un programa.
El uso de ``formal mathematical constructs'' eliminaría estas ambigüedades, aunque la formalización de la semántica para la totalidad del lenguaje C no es una tarea facil y de hecho es una que ha sido objeto de mucha investigación en el área de la semántica.

A pesar de que la semántica definida en el presente trabajo cubre un subconjunto limitado del lenguaje C, es lo suficientemente expresiva como para permitir la implementación de algorítmos tales como algorítmos de ordenamiento, búsqueda en arboles, etc.

Otro de los objetivos de este trabajo es hacer que la semántica formal sea ejecutable en el ambiente de Isabelle/HOL.
La semántica formalizada es ``deterministic'', lo que permite la definición de un interpretador que pueda, efectivamente, ejecutar la semántica.
Este interpretador retornará el estado final resultante de la ejecución de la semántica.

La generación de código C que pueda ser ejecutable está entre los objetivos planteados en este trabajo.
La semántica formal definida en este trabajo corresponde a la semántica del lenguaje C implementada por un compilador.
Se proporciona un mecanismo mediante el cual se pueden traducir los programas escritos en la semántica formal a código C que puede ser compilado y ejecutado en una máquina.
El código generado, al ser compilado y ejecutado en una máquina, presentará el mismo comportamiento que el programa interpretado dentro del ambiente de Isabelle/HOL.
Esto permite la implementación de algunos algoritmos verificados utilizando la semántica y la generación de código C eficiente a partir de la misma que puede ser compilado y ejecutado en la máquina.

Finalmente, también es objetivo de este trabajo verificar que la semántica sea compatible con un compilador de C real.
Para ello se define un arnés de pruebas y una \textit{suite} de pruebas que tienen el propósito de aumentar la confianza en el proceso de traducción, es decir, la semántica del programa no es cambiada por el proceso de traducción a lenguaje C.
El proceso de``\textit{testing}'' intenta comprobar que el estado final de un programa ejecutado en el ambiente de Isabelle/HOL y el estado final del programa generado, que es compilado y ejecutado fuera de este ambiente, serán iguales (excepto para el caso en el que se presente una falla al intentar asignar memoria dinámica).
Para garantizar esto, se escribe una librería para el arnés de pruebas en C que se utiliza para realizar pruebas generadas automáticamente (para los programas escritos en la semántica) que se encargan de comparar el estado final de la semántica ejecutada en el ambiente Isabelle/HOL con aquel del programa compilado.


\section{Marco Teórico}
\begin{comment}
FIXME! no se si está bien este título.
\end{comment}

\subsection{Semántica de un lenguaje de programción}

En esta sección se da una breve introducción a los conceptos relacionados a semántica con los que el lector debe estar familiarizado para leer el contenido de este trabajo.

\subsubsection*{Definición}

La semántica de un lenguaje de programación es el significado de programas en ese lenguaje.
Según Tennent~\cite{tennent}, para poder definir y respaldar el significado de un programa en un lenguaje de programación, se necesita una teoría matemática de la semántica de los lenguajes de programación que sea rigurosa.

Como es señalado por Nielson y Nielson~\cite{nielson}, el carácter riguroso de este estudio se debe al hecho de que puede revelar ambigüedades o complejidades subyacientes en los documentos definidos en lenguaje natural, y también que este rigor matemático es necesario para pruebas de correctitud.
Para muchos lenguajes de programación grandes, por ejemplo el lenguaje C, su documento de referencia (donde la semántica del lenguaje se explica) se encuentra escrito en lenguaje natural.
Dada la ambigüedad presente en el lenguaje natural, esto puede llevar a dificultades cuando se intenta razonar sobre los programas escritos en esos lenguajes de programación.

La falta de una semántica definida matemáticamente de una forma rigurosa hace que sea difícil para los desarrolladores escribir herramientas precisas y correctas para el lenguaje.
Debido a las ambigüedades, parte del comportamiento del lenguaje está sujeto a la interpretación del lector.
Mediante el uso de términos definidos matemáticamente podemos eliminar esta posible ambigüedad.
Si cada término se describe matemáticamente, entonces podemos asegurar que el significado definido en la semántica de un lenguaje solo puede ser uno y no puede tener diferentes interpretaciones.

Con el fin de aclarar la definición y los diferentes tipos de semántica, se presenta un ejemplo considerado relevante de Nielson y Nielson ~\cite{nielson}.
Se toma el siguiente programa
\begin{equation*}
z:=x; x:=y; y:=z
\end{equation*}
donde ``$:=$'' es una asignación a una variable y ``$;$'' es la secuenciación de instrucciones.
Sintácticamente este programa está compuesto por tres instrucciones separadas por ``$;$'', donde cada instrucción está compuesta por una variable, el símbolo ``$:=$'' y una segunda variable.

La \textit{semántica} de este programa es el significado del mismo.
Semánticamente, este programa intercambia los valores de $x$ e $y$ (usando $z$ como una variable temporal).

\subsubsection*{Tipos de semántica}

En la sección anterior, se presento un programa ejemplo y una explicación aproximada de su significado en lenguaje natural.
Esta explicación se podría haber hecho con mayor claridad y rigor al explicar formalmente el significado de las instrucciones, especialmente el significado de las instrucciones de asignación y secuenciación.

Existen muchos enfoques diferentes sobre cómo formalizar la semántica de un lenguaje de programación, dependiendo de la finalidad.
A continuación se presentan los enfoques más utilizados:


\subsubsection*{Semántica operacional}

Una semántica se define utilizando el enfoque operacional cuando el foco se pone en \textit{cómo} se ejecuta un programa.
Se puede considerar como una abstracción de la ejecución del programa en una máquina~\cite{nielson}.
Dado un programa, su explicacion operacional representa cómo se ejecuta el mismo dado un estado inicial.

Tomando el ejemplo dado anteriormente, dar una interpretación de semántica operacional para ese programa se reduce a definir cómo las instrucciones de asignación y secuenciación se ejecutan.
En un primer enfoque intuitivo se pueden distinguir dos reglas básicas:

\begin{itemize}
\item{Para ejecutar una secuencia de instrucciones, cada instrucción se ejecuta en un orden de izquierda a derecha.}
\item{Para ejecutar una instrucción de asignación entre dos variables, el valor de la variable del lado derecho se determina y se asigna a la variable del lado izquierdo.}
\end{itemize}

Existen dos tipos diferentes de semánticas operacionales: \textit{semántica de pasos cortos} (o semántica operacional estructurada) y \textit{semántica de pasos largos} (o semántica natural).
Se procederá a introducir ambos conceptos y a construir una interpretación para el ejemplo dado anteriormente haciendo uso de ambas semánticas.


\paragraph{Semántica de pasos largos}

Este tipo de semántica representa la ejecución de un programa desde un estado inicial hasta un estado final en un solo paso, por lo tanto, no permite la inspección explícita de estados de ejecución intermedios~\cite{nipkow}.

Suponiendo que se tiene un estado donde la variable $x$ tiene el valor $5$, la variable $y$ tiene el valor $7$ y la variable $z$ tiene el valor $0$ y el programa del ejemplo presentado anteriormente, la ejecución del programa completo se verá de la siguiente manera:

\begin{equation*}
\langle z:=x; x:=y; y:=z, s_{0} \rangle \rightarrow s_{3}
\end{equation*}

donde se utilizan las siguientes abreviaciones:
\begin{align*}
s_{0} &= [x\mapsto5, y\mapsto7, z\mapsto0]\\
s_{3} &= [x\mapsto7, y\mapsto5, z\mapsto5]
\end{align*}

Sin embargo, podemos obtener la siguiente ``secuencia de derivación'' para el programa anterior:

\begin{equation*}
\cfrac{
  \cfrac{\langle z:=x, s_{0}\rangle \rightarrow s_{1} \qquad \langle x:=y, s_{1} \rangle \rightarrow s_{2}}
    {\langle z:=x; x:=y, s_{0} \rangle \rightarrow s_{2}}
  \qquad
  \langle y:=z, s_{2} \rangle \rightarrow s_{3}
  }
  {\langle z:=x; x:=y; y:=z, s_{0} \rangle \rightarrow s_{3}}
\end{equation*}

donde se utilizan las siguientes abreviaciones:
\begin{align*}
s_{0} &= [x\mapsto5, y\mapsto7, z\mapsto0]\\
s_{1} &= [x\mapsto5, y\mapsto7, z\mapsto5]\\
s_{2} &= [x\mapsto7, y\mapsto7, z\mapsto5]\\
s_{3} &= [x\mapsto7, y\mapsto5, z\mapsto5]
\end{align*}

La ejecución de $z:=x$ en el estado $s_{0}$ producirá el estado $s_{1}$ y la ejecución de $x:=y$ en el estado $s_{1}$ producirá el estado $s_{2}$.
Por lo tanto la ejecución de $z:=x; x:=y$ en el estado $s_{0}$ producirá el estado $s_{2}$.
Además, la ejecución de $y:=z$ en el estado $s_{2}$ producirá el estado $s_{3}$.
Finalemnte, la ejecución de todo el programa $z:=x; x:=y; y:=z$ en el estado $s_{0}$ producirá el estado $s_{3}$.

\paragraph{Semántica de pasos cortos}

A veces es deseable tener mayor información con respecto a los estados intermedios de un programa, es por eso que la semántica de pasos cortos existe.

Este tipo de semántica representa pequeños pasos de ejecución atómicos en un programa y permite razonar sobre qué tanto ha sido ejecutado de un programa y explicitamente inspeccionar ejecuciones parciales~\cite{nipkow}

En este ejemplo se comienza desde el programa completo y se toman pequeños pasos (denotados por ``$\Rightarrow$'') que produce el resto del programa que queda por ejecutar luego de ejecutar un paso corto y el estado resultante luego de ejecutar el mismo, hasta que todo el programa es ejecutado.

Suponiendo que se tiene un estado donde la variable $x$ tiene el valor $5$, la variable $y$ tiene el valor $7$ y la variable $z$ tiene el valor $0$¸ y el programa del ejemplo, se obtiene la siguiente ``secuencia de derivación'':

\begin{equation*}
\begin{split}
& \phantom{\Rightarrow} \phantom{=} \langle z:=x; x:=y; y:=z, [x\mapsto5, y\mapsto7, z\mapsto0]\rangle\\
& \Rightarrow \phantom{=} \phantom{z:=x} \langle x:=y; y:=z [x\mapsto5, y\mapsto7, z\mapsto5]\rangle\\
& \Rightarrow \phantom{=} \phantom{z:=x; x:=y} \langle y:=z [x\mapsto7, y\mapsto7, z\mapsto5]\rangle\\
& \Rightarrow \phantom{=} \phantom{z:=x; x:=y; y:=z} [x\mapsto7, y\mapsto5, z\mapsto5]
\end{split}
\end{equation*}

Lo que sucede aquí es que en en primer paso, la instrucción $z:=x$ se ejecuta y el valor de la variable $z$ cambia a $5$, $x$ e $y$ permanecen sin cambios.
Luego de la ejecución de la primera instrucción, queda el programa $x:=y; y:=z$.
Se ejecuta la segunda instrucción $x:=y$ y el valor de $x$ cambia a $7$, $y$ y $z$ permanecen sin cambios.
Entonces, queda el programa $y:=z$, despues de ejecutar esta instrucción final, el valor de $y$ cambia a $5$, $x$ y $z$ permanecen sin cambios.

Por último, se tiene que el comportamiento de este programa es intercambiar los valores de $x$ e $y$ utilizando $z$ como una variable temporal.


\subsubsection*{Semántica denotacional}
La semántica denotacional deja de centrarse en \textit{como} se ejecuta un programa y redirige su enfoce hacia el \textit{efecto} de ejecutar el programa.
Este enfoque sirve de ayuda pues da un \textit{significado} a los programas en un lenguaje~\cite{nipkow}.
Este enfoque se modela mediante el uso de funciones matemáticas.
Se tiene una función por cada ``construcción'' en el lenguaje, que define su significado, y estas funciones operan sobre estados.
Toman el estado inicial y produce el estado resultante de aplicar el efecto de la ``construcción''.
\begin{comment}
FIXME Construcción no es la palabra.
\end{comment}

Si se toma el ejemplo anterior, se pueden definir los efectos de las diferentes ``construcciones'' que tenemos: instrucciones de secuenciación y asignación.

\begin{itemize}
\item{El efecto de una secuencia de instrucciones se define como la composición funcional de cada instrucción individual.}
\item{El efecto de una asignación entre dos variables se define como una función que toma un estado y produce el mismo estado, donde el valor actual de la variable del lado izquierdo es actualizada con el nuevo valor de la variable del lado derecho.}
\end{itemize}

Para este ejemplo en particular se obtendrían funciones de la forma $S [\![ z:=x ]\!]$, $S [\![ x:=y ]\!]$ and $S [\![ y:=z ]\!]$ para cada instrucción individual.
Por otra parte, para la instrucción compuesta que es el programa completo, se obtendría la siguiente función:

\begin{equation*}
S [\![ z:=x; x:=y; y:=z ]\!] = S [\![ y:=z ]\!] \circ S [\![ x:=y ]\!] \circ S [\![ z:=x ]\!]
\end{equation*}

La ejecución del programa completo $z:=x; x:=y, y:=z$ tendría el efecto de \textit{aplicar} la función $S [\![ z:=x; x:=y; y:=z ]\!]$ al estado inicial $[x\mapsto5, y\mapsto7, z\mapsto0]$:

\begin{align*}
S [\![ z:=x; & x:=y; y:=z ]\!]([x\mapsto5, y\mapsto7, z\mapsto0])\\
&= (S [\![ y:=z ]\!] \circ S [\![ x:=y ]\!] \circ S [\![ z:=x ]\!])([x\mapsto5, y\mapsto7, z\mapsto0])\\
&= S [\![ y:=z ]\!](S [\![ x:=y ]\!] (S [\![ z:=x ]\!]))([x\mapsto5, y\mapsto7, z\mapsto0])\\
&= S [\![ x:=y ]\!] (S [\![ z:=x ]\!])([x\mapsto5, y\mapsto7, z\mapsto5])\\
&= S [\![ z:=x ]\!]([x\mapsto7, y\mapsto7, z\mapsto5])\\
&= [x\mapsto7, y\mapsto5, z\mapsto5]
\end{align*}

El enfoque está en el estado resultante que representa el efecto que el programa tuvo en un estado inicial.
Es más sencillo razonar sobre los programas utilizando este enfoque ya que es similar a razonar sobre objetos matemáticos.
Aunque es importante tomar en cuenta que el establecimiento de una base matemática firme para hacerlo no es una tarea trivial.
El enfoque denotacional puede ser facilmente adaptado para representar algunas propiedades de los programas.
Ejemplos de esto son inicialización de variables, plegamiento de constantes (\textit{constant folding}) y ``accesabilidad''~\cite{nielson}.


\subsubsection*{Semántica Axiomática}

\begin{comment}
FIXME construcción
\end{comment}
Tambien conocida como Lógica de Hoare, este enfoque final se utiliza cuando el interés recae en probar propiedades de los programas.
Se puede hablar de \textit{correctitud parcial} de un programa con respecto a una ``construcción'', una precondición y una postcondición siempre que la siguiente implicación se cumpla:

\begin{displayquote}
Si la precondición se cumple antes de que el programa se ejecute, y la ejecución del programa termina, entonces la postcondición se cumple para el estado final.
\end{displayquote}

También se puede hablar de \textit{correctitud total} de un programa con respecto a una ``construcción'', una precondición y una postcondición siempre que la siguiente implicación se cumpla:

\begin{displayquote}
Si la precondición se cumple antes de que el programa se ejecute y la ejecución del programa termina, entonces la postcondición se cumple para el estado final.
\end{displayquote}

Por lo general es mas sencillo hablar sobre el concepto de correctitud parcial~\cite{nipkow}.


La siguiente propiedad de correctitud parcial se define para el programa del ejemplo:

\begin{equation*}
\lbrace x=n \land y=m \rbrace z:=x; x:=y; y:=z \lbrace x=m \land y=n \rbrace
\end{equation*}

donde $\lbrace x=n \land y=m \rbrace $ y $\lbrace x=m \land y=n \rbrace $ son la precondición y postcondición respectivamente.
$n$ y $m$ indican los valores iniciales de $x$ e $y$.
El estado $[x\mapsto5, y\mapsto7, z\mapsto0]$ cumple la precondición si se toman $n=5$ y $m=7$.
Luego de que la propiedad de correctitud parcial es \textit{demostrada}, entonces se puede deducir que \textit{si} el programa termina \textit{entonces} lo hará en un estado donde $x$ es $7$ e $y$ es $5$.

Este enfoque se basa en un \textit{sistema de demostración} o reglas de inferencia para derivar y demostrar propiedades de correctitud parcial~\cite{nipkow}.
El siguiente ``arbol de demostración'' puede expresar una prueba de la propiedad de correctitud parcial anterior.
\begin{comment}
FIXME demostrar o probar
\end{comment}

\begin{equation*}
\cfrac{
  \cfrac{ \lbrace p_{0} \rbrace z:=x \lbrace p_{1} \rbrace \qquad \lbrace p_{1} \rbrace x:=y \lbrace p_{2} \rbrace }
    { \lbrace p_{0} \rbrace z:=x; x:=y \lbrace p_{2} \rbrace }
  \qquad
 \lbrace p_{2} \rbrace y:=z \lbrace p_{3} \rbrace
  }
  { \lbrace p_{0} \rbrace z:=x; x:=y; y:=z \lbrace p_{3} \rbrace }
\end{equation*}

donde se utilizan las siguientes abreviaciones:
\begin{align*}
p_{0} &= x=n \land y=m\\
p_{1} &= y=m \land z=n\\
p_{2} &= x=m \land z=n\\
p_{3} &= x=m \land y=n
\end{align*}

La ventaja de usar este enfoque es que se cuenta con una manera fácil de demostrar propiedades de un programa, dada por el sistema de demostración.


\subsection{Isabelle/HOL}\label{section:isabelle/hol}

Hoy en día se cuenta con la existencia de demostradores de teoremas automatizados que ayudan en la formalización y demostración de diferentes programas.
Las pruebas escritas en papel y lápiz son propensas a errores y los seres humanos son mas fáciles de engañar que a una máquina.
Por lo tanto se deben aprovechar los recursos ofrecidos por una máquina para permitir que realice el trabajo pesado.
Razonar sobre la semántica de un lenguaje de programación sin el uso de herramientas automátizadas se convierte en una gran tarea y, como se dijo antes, propensa a errores, por no hablar de que la certeza sobre la correctitud de las demostraciones disminuye.

Mediante el uso de un demostrador de teoremas, en el caso de este trabajo Isabelle/HOL~\cite{isabelle-tutorial}, se puede estar seguro de que nos resultados demostrados son correctos.
En el entorno de Isabelle/HOL se pueden hacer definiciones lógicas y demostrar lemmas y teoremas sobre esas definiciones de una manera sólida.
La semántica para el lenguaje utilizado en este trabajo está formalizada en Isabelle/HOL, así como las pruebas que acompañan a estas definiciones.
Las definiciones de la semántica formal y pruebas que lo acompañan están escritas en Isabelle/HOL.
No se especificarán los detalles de Isabelle/HOL en esta sección sino que se remite al lector al libro \textit{Concrete Semantics with Isabelle/HOL}~\cite{nipkow} en el que se muestra una introducción a Isabelle/HOL y demotración de teoremas.

Por otra parte, Isabelle/HOL también cuenta con herramientas de generación de código~\cite{isabelle-codegen} que permiten la generación de código ejecutable en SML correspondiente a la especificación HOL de la semántica.
Este código generado permitirá mas adelante la ejecución de la semántica definida.
Tambien, Isabelle/HOL permite que código escrito en Isabelle/ML esté embebido en las teorías~\cite{isabelle-implementation}, lo cual facilita el proceso de traducción a lenguaje C.

\subsection{Chloe}\label{subsection:chloe}

En el presente trabajo se formaliza la semántica de un pequeño lenguaje de programación llamado Chloe.
Este lenguaje es un subconjunto del lenguaje de programación C.
Aunque la sintaxis y semántica de este lenguaje se presentan en mayor detalle en el capítulo~\ref{chapter:semantics} es importante mencionar que se trata de una semántica operacional de pasos cortos.

Este lenguaje tiene las siguientes características: variables, arreglos, aritmética de apuntadores, ciclos, condicionales, funciones y memoria dinámica.
El alcance de este proyecto se limitó a las características mencionadas anteriormente y hay varias características que actualmente no son soportadas por el lenguaje.
Estas características son: un sistema estático de tipos que este probado que sea correcto y sólido, concurrencia, operaciones I/O, goto, etiquetas, break y continue.

Si bien será conveniente en el fururo tener las características que actualmente no cubre el alcance de este trabajo, el actual conjunto de características soportadas por Chloe son suficientes para mostrar ejemplos relevantes de programas y tiene suficiente poder expresivo como para ser Turing-completo.


\section{Estructura del documento}\label{section:document_structure}

El resto de este documento se divide de la siguiente manera: el capítulo~\ref{chapter:previous} abarca los trabajos previos y relacionados al presente, el capítulo~\ref{chapter:semantics} abarca los detalles de la sintaxis y la semántica de pasos cortos definida para Chloe, el capítulo~\ref{chapter:pretty} abarca el proceso de traducción de un programa en la semántica de Chloa a un programa en C, a este proceso se le llama \textit{pretty printing}, el capítulo~\ref{chapter:testing} abarca el conjunto de pruebas que verifican la correctitud del proceso de traducción y finalmente, el capítulo~\ref{chapter:conclusion} encapsula el resultado del trabajo y las conclusiones finales del mismo, asi como también detalla la dirección a seguir para realizar trabajo a futuro a partir del presente.
