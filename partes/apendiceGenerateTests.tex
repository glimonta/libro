\chapter{Generación de código con pruebas}
\label{ap:generate_c_test_code}
\lhead{Apéndice L. \emph{Generación de código con pruebas}}

\begin{figure}
\begin{lstlisting}[mathescape=true]
  fun generate_c_test_code (SOME (code,test_code)) rel_path name thy =
    let
      val code = code |> String.implode
      val test_code = test_code |> String.implode
    in
      if rel_path="" orelse name="" then
        (writeln (code ^ " <rem last line> " ^ test_code); thy)
      else let
        val base_path = Resources.master_directory thy
        val rel_path = Path.explode rel_path
        val name_path = Path.basic name |> Path.ext "c"

        val abs_path = Path.appends [base_path, rel_path, name_path]
        val abs_path = Path.implode abs_path

        val _ = writeln ("Writing to file " ^ abs_path)

        val os = TextIO.openOut abs_path;
        val _ = TextIO.output (os, code);
        val _ = TextIO.flushOut os;
        val _ = TextIO.closeOut os;

        val _ = Isabelle_System.bash ("sed -i '$\mathtt{\$}$d ' " ^ abs_path);

        val os = TextIO.openAppend abs_path;
        val _ = TextIO.output (os, test_code);
        val _ = TextIO.flushOut os;
        val _ = TextIO.closeOut os;
      in thy end
    end

  | generate_c_test_code NONE _ _ _ =
      error "Invalid program or failed execution"

  fun expect_failed_test (SOME _) = error "Expected Failed test"
    | expect_failed_test NONE = ()
\end{lstlisting}
\caption{Generación de código en C con pruebas}
\end{figure}

Esta función toma cuatro parámetros.
El primer parámetro de la función es de tipo \verb|(string, string) option| que corresponde a la tupla que contiene la representación en cadena de caracteres del programa en código C y las pruebas para ese programa, respectivamente.
Si la función recibe un valor \verb|None|, significa que la función \verb|prepare_test_export| falló y no se debe generar un programa en C.
El segundo parámetro es la ruta al directorio al que se quiere exportar el programa.
Este parámetro es una ruta relativa al directorio donde se encuentra la teoría de Isabelle que contiene las directivas para hacer \textit{pretty printing}.
El tercer parámetro es el nombre del programa.
El último parámetro es el contexto de la teoría actual.

Si la ruta dada es una cadena de caracteres vacía el código generado se imprimirá a la vista de salida de Isabelle.
Esta función crea un nuevo archivo con el nombre indicado en los parámetros con ``\verb|.c|'' agregado al final (es decir, $<nombre>\mathtt{.c}$) en el directorio indicado por el parámetro de la ruta.
El usuario puede entonces encontrar el código generado en el directorio indicado.
