\chapter{Demostraciones de lemas de la semántica}
\label{ap:determinism}
\lhead{Apéndice C. \emph{Demostraciones de lemas de la semántica}}


Cada vez que se introduce un lema, junto a su número se encontrará un nombre en paréntesis.
Este nombre corresponde al nombre del lema en el código fuente de Isabelle presentados con este trabajo.
En el caso en que el lector esté interesado en la demostración exacta para un lema en particular él o ella puede buscar el lema correspondiente en los archivos y leer la prueba.

Este primer lema indica que cada vez que se tiene un comando diferente a \verb|SKIP| hay siempre una arista habilitada (la función \textit{enabled} retorna \verb|True|) que se puede tomar en el CFG o un error ocurre al evaluar la función \textit{enabled}.

\begin{lemma}[cfg\_has\_enabled\_action]
$\newline$
$c$ $\neq$ \verb|SKIP| $\Longrightarrow$ $\exists\ c'\ en\ tr$. \verb|cfg| $c\ (en,tr)\ c'\ \wedge\ (en\ s\ =$ \verb|None| $\vee\ en\ s\ =$ \verb|Some True|$)$
\label{lemma:cfg_enabled_action}
\end{lemma}

\begin{proof}
La prueba es por inducción en el comando.
A excepción de dos casos, la prueba es demostrada automáticamente.
Esos dos casos son los comandos \verb|Seq| y \verb|If|.
En el caso de \verb|Seq| se debe hacer una prueba por casos sobre el primer comando de la secuenciación para poder diferenciar los casos cuando es \verb|SKIP| y cuando no.
En ambos casos existe una regla en la definición de CFG que asegura que hay una arista habilitada que se puede tomar y el caso se soluciona automáticamente.

En el caso de \verb|If| se debe hacer una prueba por casos sobre el valor de $\mathtt{en\_pos}\ b\ s$.
Puede fallar y retornar un valor \verb|None|, entonces el caso está resuelto.
Si no falla entonces se debe chequear el valor retornado por la función \verb|en_pos|.
En el caso donde es \verb|True| entonces el caso está resuelto.
El caso difícil es cuando la evaluación de \verb|en_pos| es \verb|False|, semánticamente, esto significa que se debe tomar la rama del \verb|else|.
Por lo tanto cuando $\mathtt{en\_pos}\ b\ s$ es \verb|False|, $\mathtt{en\_neg}\ b\ s$ siempre se evaluará como \verb|True|, esto significa que el comando siempre tendrá una arista habilitada que puede seguir, es decir, $\mathtt{en\_neg}\ b\ s$ y la demostración de este caso está completa.
\end{proof}

A continuación se quiere demostrar que mientras la pila no esté vacía, la semántica de pasos cortos siempre puede tomar un paso.
Se utilizará el lema anterior en la prueba para este nuevo lema.
Este lema dice que la semántica puede tomar un paso.

\begin{lemma}[can\_take\_step]
$\newline$
$\neg$ \verb|is_empty_stack| $s$ $\Longrightarrow$ $\exists\ x.\ s\ \rightarrow\ x$
\label{lemma:can_take_step}
\end{lemma}

\begin{proof}
De las suposiciones sabemos que la pila no está vacía, por lo tanto se puede reescribir el estado de la siguiente manera: $s = ((c, locals, rloc)\#\sigma,\gamma,\mu)$.
Luego se puede hacer una prueba por casos.
Se tiene el caso donde $c\ =\ \mathtt{SKIP}$ y el caso donde $c\\neq\ \mathtt{SKIP}$.

El caso donde $c\ =\ \mathtt{SKIP}$ es el caso en el que se está retornando de la ejecución de una función.
Se tiene que la pila no está vacía, que $c\ =\ \mathtt{SKIP}$ y que $s = ((c, locals, rloc)\#\sigma,\gamma,\mu)$.
Este caso corresponde a las reglas \verb|Return_void| y \verb|Return_void_None| en la definición de \verb|small_step|.
En ambos casos la semántica puede tomar un paso, bien sea a un nuevo estado $s'$ o a \verb|None|.
El caso se resuelve automáticamente por Isabelle al indicar las reglas mencionadas y las suposiciones.

El segundo caso es cuando $c\\neq\ \mathtt{SKIP}$.
Este es el caso donde se ejecuta cualquier otro comando que no sea el retorno de una función.
En este caso se utiliza el lema anterior~\ref{lemma:cfg_enabled_action} y se tiene que $\mathtt{cfg}\ c\ (en,tr)\ c'$ y o la función \textit{enabled} falla ($\mathtt{en}\ s\ = \mathtt{None}$) o está habilitada ($\mathtt{en}|\ s\ = \mathtt{Some True}$).

Esto se demuestra mediante una separación por casos.
En el primer caso se supone $\mathtt{cfg}\ c\ (en,tr)\ c'$ y $\mathtt{en}\ s\ = \mathtt{None}$.
Este es el caso donde la evaluación de la función falla, se sabe que este caso toma un paso corto a \verb|None| dado por el caso \verb|None| de la definición de \verb|small_step|.
Entonces se ha demostrado que existe un estado tal que $s\ \rightarrow\ \mathtt{None}$.

En el siguiente caso se supone $\mathtt{cfg}\ c\ (en,tr)\ c'$ y $\mathtt{en}\ s\ = \mathtt{Some True}$.
Este es el caso donde hay una arista que está habilitada para ser seguida y aún se deben revisar dos casos mas.
Aplicar la función \textit{transformer} sobre el estado actualizado con el comando ($tr\ (upd\_com\ c'\ s)$) puede fallar o no.
En el caso donde falla (retorna \verb|None|) se puede tomar un paso a \verb|None| y se habrá probado que hay un estado al que $s$ puede tomar un paso corto, es decir \verb|None|.
En el caso donde no falla, retornará $\mathtt{Some}\ s_{2}$.
En este caso se puede tomar un paso a $\mathtt{Some}\ s_{2}$ y se habrá probado que hay un estado al cual $s$ puede tomar un paso corto, es decir $\mathtt{Some}\ s_{2}$.
\end{proof}

Se definen varios lemas que serán útiles más adelante.

El primer lema establece que el CFG se queda atascado en SKIP:
\begin{lemma}[cfg\_SKIP\_stuck]
$\newline$
$\neg$ \verb|cfg SKIP| $a\ c$
\label{lemma:stuck_at_skip}
\end{lemma}

\begin{proof}
La propiedad es demostrada automáticamente.
\end{proof}


\begin{lemma}[ss\_empty\_stack\_stuck]
$\newline$
\verb|is_empty_stack| $s$ $\Longrightarrow$ $\neg$ $(s\ \rightarrow\ cs')$
\label{lemma:ss_empty_stack_stuck}
\end{lemma}

\begin{proof}
La propiedad es demostrada automáticamente.
\end{proof}


\begin{lemma}[ss'\_SKIP\_stuck]
$\newline$
\verb|is_empty_stack| $s$ $\Longrightarrow$ $\neg$ $(Some s\ \rightarrow\ cs')$
\label{lemma:ss'_empty_stack_stuck}
\end{lemma}

\begin{proof}
La propiedad es demostrada automáticamente.
\end{proof}

Los lemas~\ref{lemma:ss_empty_stack_stuck} y~\ref{lemma:ss'_empty_stack_stuck} definen el estado final en el que la semántica se quedará atascada, la semántica se quedará atascada cuando no existan mas marcos de pila en la pila.


\begin{lemma}[en\_neg\_by\_pos]
$\newline$
\verb|en_neg| $e\ s$ = \verb|map_option (HOL.Not) (en_pos| $e\ s$ \verb|)|
\label{lemma:en_neg_by_pos}
\end{lemma}

\begin{proof}
La propiedad es demostrada automáticamente desplegando las definiciones de \verb|en_neg| y \verb|en_pos|.
\end{proof}

El lema~\ref{lemma:en_neg_by_pos} establece que cada vez que la función \verb|en_neg| tenga un valor (diferente de None) sobre un estado, \verb|en_pos| tendrá el mismo resultado opuesto.
En el caso donde una de las funciones falla, la otra fallará también y retornarán \verb|None|.
Si no fallan, sus resultados serán opuestos, esto es, si una tiene \verb|Some True| como resultado, la otra tendrá \verb|Some False| como resultado.
Este lema será útil al probar que la semántica es determinística.

\begin{lemma}[cfg\_determ]
$\newline$
$\mathtt{cfg}$ $c$ $a1$ $c'$ $\wedge$
$\mathtt{cfg}$ $c$ $a2$ $c''$
$\newline$
$\Longrightarrow$
$a1\ =\ a2$ $\wedge$ $c'\ =\ c''$ $\vee$
$\newline$
$\exists\ b.\ a1\ = (\mathtt{en\_pos}\ b,\ \mathtt{tr\_eval}\ b)\ \wedge\ a2\ = (\mathtt{en\_neg}\ b,\ \mathtt{tr\_eval}\ b)\ \vee$
$\newline$
$\exists\ b.\ a1\ = (\mathtt{en\_neg}\ b,\ \mathtt{tr\_eval}\ b)\ \wedge\ a2\ = (\mathtt{en\_pos}\ b,\ \mathtt{tr\_eval}\ b)$
\label{lemma:cfg_determ}
\end{lemma}

\begin{proof}
La demostración es por inducción sobre el comando, con los casos generados automáticamente por Isabelle de las reglas de \verb|cfg|.
Todos los casos se demuestran automáticamente.
\end{proof}

El lema~\ref{lemma:cfg_determ} establece que CFG es determinístico.
El único caso donde no lo es es en el caso del condicional, para el cual se agrega una alternativa extra en la conclusión.
Puede pasar que se tenga una regla del CFG comenzando en un comando \verb|If| que tiene una arista a $c_{1}$ y también una arista a $c_{2}$.
Esto no es realmente un problema dado que las funciones \textit{enabled} garantizan que cuando esto ocurre, solamente una de las aristas puede ser tomada.


\begin{lemma}[lift\_upd\_com]
$\newline$
$\neg\ \mathtt{is\_empty\_stack}\ s\ \Longrightarrow$
$\newline$
$\mathtt{lift\_transformer\_nr}\ tr\ (\mathtt{upd\_com}\ c\ s)\ =$
$\newline$
$\mathtt{map\_option}\ (\mathtt{upd\_com}\ c)\ (\mathtt{lift\_transformer\_nr}\ tr\ s)$
\label{lemma:lift_upd_com}
\end{lemma}

\begin{proof}
Es demostrado automáticamente desplegando la definición de \verb|lift_transformer_nr|.
\end{proof}

\begin{lemma}[tr\_eval\_upd\_com]
$\newline$
$\neg\ \mathtt{is\_empty\_stack}\ s\ \Longrightarrow$
$\newline$
$\mathtt{tr\_eval}\ e\ (\mathtt{upd\_com}\ c\ s)\ =$
$\newline$
$\mathtt{map\_option}\ (\mathtt{upd\_com}\ c)\ (\mathtt{tr\_eval}\ e\ s)$
\label{lemma:tr_eval_upd_com}
\end{lemma}

\begin{proof}
Es demostrado automáticamente desplegando la definición de \verb|tr_eval|.
\end{proof}

La función \verb|lift_transformer_nr| levanta la definición de una función \textit{transformer}, que opera en el nivel de estados visibles, a estados.
El lema~\ref{lemma:lift_upd_com} establece que no importa en que orden se aplique una función \textit{transformer} sobre un estado y se actualice el comando en el tope de la pila, dado que siempre produce el mismo resultado.
Esto es porque la función que actualiza el comando solo modifica el comando en el tope de la pila y la función \textit{transformer} no puede acceder y modificar esa parte de la pila, ya que solo puede modificar el estado visible.

El lema~\ref{lemma:tr_eval_upd_com} es una versión más específica del lema anterior que establece el mismo hecho específicamente para la función \verb|tr_eval|.


Todos los lemas y definiciones anteriores son definidos con el fin de ser utilizados en la próxima demostración; la semántica de pasos cortos es determinística.
\begin{lemma}[small\_step\_determ]
$\newline$
$s\ \rightarrow\ s'\ \wedge\ s\ \rightarrow\ s''\ \Longrightarrow\ s'\ =\ s''$
\label{lemma:small_step_determ}
\end{lemma}

\begin{proof}
La demostración es una prueba por casos sobre la semántica de pasos cortos.
Se obtienen 4 casos, cada uno correspondiente a cada regla en la semántica de pasos cortos.
Los objetivos de prueba generados por las reglas \verb|Return_void| y \verb|Return_void_None| se resuelven automáticamente.
Los objetivos de prueba generados por las reglas \verb|Base| y \verb|None| se resuelven automáticamente luego de agregar los lemas~\ref{lemma:en_neg_by_pos} y ~\ref{lemma:tr_eval_upd_com}.
\end{proof}

Luego solo queda demostrar que \verb|small_step'| es también determinística.

\begin{lemma}[small\_step'\_determ]
$\newline$
$s\ \rightarrow'\ s'\ \wedge\ s\ \rightarrow'\ s''\ \Longrightarrow\ s'\ =\ s''$
\label{lemma:small_step'_determ}
\end{lemma}

\begin{proof}
La demostración es una prueba por casos sobre la semántica de pasos cortos.
Se demuestra automáticamente mediante el uso del lema~\ref{lemma:small_step_determ}.
\end{proof}

