\chapter{Demostraciones de lemas de la semántica}
\label{ap:determinism}
\lhead{Apéndice C. \emph{Demostraciones de lemas de la semántica}}

Demostración del lema~\ref{lemma:cfg_enabled_action}:
\begin{lemmano}[cfg\_has\_enabled\_action]
$\newline$
$c$ $\neq$ \verb|SKIP| $\Longrightarrow$ $\exists\ c'\ en\ tr$. \verb|cfg| $c\ (en,tr)\ c'\ \wedge\ (en\ s\ =$ \verb|None| $\vee\ en\ s\ =$ \verb|Some True|$)$
\end{lemmano}

\begin{proof}
La prueba es por inducción en el comando.
A excepción de dos casos, la prueba es demostrada automáticamente.
Esos dos casos son los comandos \verb|Seq| y \verb|If|.
En el caso de \verb|Seq| se debe hacer una prueba por casos sobre el primer comando de la secuenciación para poder diferenciar los casos cuando es \verb|SKIP| y cuando no.
En ambos casos existe una regla en la defición de CFG que asegura que hay una arista habilitada que se puede tomar y el caso se soluciona automáticamente.

En el caso de \verb|If| se debe hacer una prueba por casos sobre el valor de $\mathtt{en\_pos}\ b\ s$.
Puede fallar y retornar un valor \verb|None|, entonces el caso está resuelto.
Si no falla entonces se debe chequear el valor booleano retornado por la función \verb|en_pos|.
En el caso donde es \verb|True| entonces el caso está resuelto.
El caso difícil es cuando la evaluación de \verb|en_pos| es \verb|False|, semanticamente, esto significa que se debe tomar la rama del else.
Por lo tanto cuando $\mathtt{en\_pos}\ b\ s$ es \verb|False|, $\mathtt{en\_neg}\ b\ s$ siempre se evaluará como \verb|True|, esto significa qe el comando siempre tendrá una arista habilitada que puede seguir, es decir, $\mathtt{en\_neg}\ b\ s$ y la demostración de este caso está completa.
\end{proof}


Demostración del lema~\ref{lemma:can_take_step}:
\begin{lemmano}[can\_take\_step]
$\newline$
$\neg$ \verb|is_empty_stack| $s$ $\Longrightarrow$ $\exists\ x.\ s\ \rightarrow\ x$
\end{lemmano}

\begin{proof}
De las supociiones sabemos que la pila no está vacia, por lo tanto se puede reescribir el estado de la siguiente manera: $s = ((c, locals, rloc)\#\sigma,\gamma,\mu)$.
Luego se puede hacer una prueba por casos.
Se tiene el caso donde $c\ =\ \mathtt{SKIP}$ y el caso donde $c\\neq\ \mathtt{SKIP}$.

El caso donde $c\ =\ \mathtt{SKIP}$ es el caso en el que se está retornando de la ejecución de una función.
Se tiene que la pila no está vacia, que $c\ =\ \mathtt{SKIP}$ y que $s = ((c, locals, rloc)\#\sigma,\gamma,\mu)$.
Este caso corresponde a las reglas \verb|Return_void| y \verb|Return_void_None| en la definición de \verb|small_step|.
En ambos casos la semántica puede tomar un paso, bien sea a un nuevo estado $s'$ o a \verb|None|.
El caso se resuelve automáticamente por Isabelle al indicar las reglas mencionadas y las suposiciones.

El segundo caso es cuando $c\\neq\ \mathtt{SKIP}$.
Este es el caso donde se ejecuta cualquier otro comando que no sea el retorno de una función.
En este caso se utiliza el lema anterior~\ref{lemma:cfg_enabled_action} y se tiene que $\mathtt{cfg}\ c\ (en,tr)\ c'$ y o la función \textit{enabled} falla ($\mathtt{en}\ s\ = \mathtt{None}$) o está habilidada ($\mathtt{en}|\ s\ = \mathtt{Some True}$).

Esto se demuestra mediante una separación por casos.
En el primer caso se supone $\mathtt{cfg}\ c\ (en,tr)\ c'$ y $\mathtt{en}\ s\ = \mathtt{None}$.
Este es el caso donde la evaluación de la función falla, se sabe que este caso toma un paso corto a None dado por el caso \verb|None| de la definción de \verb|small_step|.
Entonces se ha demostrado que existe un estado tal que $s\ \rightarrow\ \mathtt{None}$.

En el siguiente caso se supone $\mathtt{cfg}\ c\ (en,tr)\ c'$ y $\mathtt{en}\ s\ = \mathtt{Some True}$.
Este es el caso donde hay una arista que está habilitada para ser seguida y aún se deben revisar dos casos mas.
Aplicar la función \textit{transformer} sobre el estado actualizado con el comando ($tr\ (upd\_com\ c'\ s)$) puede fallar o no.
En el caso donde falla (retorna \verb|None|) se puede tomar un paso a \verb|None| y se habrá probado que hay un estado al que $s$ puede tomar un paso corto, es decir \verb|None|.
En el caso donde no falla, retornará $\mathtt{Some}\ s_{2}$.
En este caso se puede tomar un paso a $\mathtt{Some}\ s_{2}$ y se habrá probado que hay un estado al cual $s$ puede tomar un paso corto, es decir $\mathtt{Some}\ s_{2}$.
\end{proof}


Demostración del lema~\ref{lemma:stuck_at_skip}:
\begin{lemmano}[cfg\_SKIP\_stuck]
$\newline$
$\neg$ \verb|cfg SKIP| $a\ c$
\end{lemmano}

\begin{proof}
La propiedad es demostrada automáticamente.
\end{proof}


Demostración del lema~\ref{lemma:ss_empty_stack_stuck}:
\begin{lemmano}[ss\_empty\_stack\_stuck]
$\newline$
\verb|is_empty_stack| $s$ $\Longrightarrow$ $\neg$ $(s\ \rightarrow\ cs')$
\end{lemmano}

\begin{proof}
La propiedad es demostrada automáticamente.
\end{proof}


Demostración del lema~\ref{lemma:ss'_empty_stack_stuck}:
\begin{lemmano}[ss'\_SKIP\_stuck]
$\newline$
\verb|is_empty_stack| $s$ $\Longrightarrow$ $\neg$ $(Some s\ \rightarrow\ cs')$
\end{lemmano}

\begin{proof}
La propiedad es demostrada automáticamente.
\end{proof}


Demostración del lema~\ref{lemma:en_neg_by_pos}:
\begin{lemmano}[en\_neg\_by\_pos]
$\newline$
\verb|en_neg| $e\ s$ = \verb|map_option (HOL.Not) (en_pos| $e\ s$ \verb|)|
\end{lemmano}

\begin{proof}
La propiedad es demostrada automáticamente desplegando las definiciones de \verb|en_neg| y \verb|en_pos|.
\end{proof}


Demostración del lema~\ref{lemma:cfg_determ}:
\begin{lemmano}[cfg\_determ]
$\newline$
$\mathtt{cfg}$ $c$ $a1$ $c'$ $\wedge$
$\mathtt{cfg}$ $c$ $a2$ $c''$
$\newline$
$\Longrightarrow$
$a1\ =\ a2$ $\wedge$ $c'\ =\ c''$ $\vee$
$\newline$
$\exists\ b.\ a1\ = (\mathtt{en\_pos}\ b,\ \mathtt{tr\_eval}\ b)\ \wedge\ a2\ = (\mathtt{en\_neg}\ b,\ \mathtt{tr\_eval}\ b)\ \vee$
$\newline$
$\exists\ b.\ a1\ = (\mathtt{en\_neg}\ b,\ \mathtt{tr\_eval}\ b)\ \wedge\ a2\ = (\mathtt{en\_pos}\ b,\ \mathtt{tr\_eval}\ b)$
\end{lemmano}

\begin{proof}
La demostración es por inducción sobre el comando, con los casos generados automáticamente por Isabelle de las reglas de \verb|cfg|.
Todos los casos se demuestran automáticamente.
\end{proof}


Demostración del lema~\ref{lemma:lift_upd_com}:
\begin{lemmano}[lift\_upd\_com]
$\newline$
$\neg\ \mathtt{is\_empty\_stack}\ s\ \Longrightarrow$
$\newline$
$\mathtt{lift\_transformer\_nr}\ tr\ (\mathtt{upd\_com}\ c\ s)\ =$
$\newline$
$\mathtt{map\_option}\ (\mathtt{upd\_com}\ c)\ (\mathtt{lift\_transformer\_nr}\ tr\ s)$
\end{lemmano}

\begin{proof}
Es demostrado automáticamente al desplegar la definición de la función \verb|lift_transformer_nr|.
\end{proof}

Demostración del lema~\ref{lemma:tr_eval_upd_com}:
\begin{lemmano}[tr\_eval\_upd\_com]
$\newline$
$\neg\ \mathtt{is\_empty\_stack}\ s\ \Longrightarrow$
$\newline$
$\mathtt{tr\_eval}\ e\ (\mathtt{upd\_com}\ c\ s)\ =$
$\newline$
$\mathtt{map\_option}\ (\mathtt{upd\_com}\ c)\ (\mathtt{tr\_eval}\ e\ s)$
\end{lemmano}

\begin{proof}
Es demostrado automáticamente desplegando la definición de \verb|tr_eval|.
\end{proof}


Demostración del lema~\ref{lemma:small_step_determ}:
\begin{lemmano}[small\_step\_determ]
$\newline$
$s\ \rightarrow\ s'\ \wedge\ s\ \rightarrow\ s''\ \Longrightarrow\ s'\ =\ s''$
\end{lemmano}

\begin{proof}
La demostración es una prueba por casos sobre la semántica de pasos cortos.
Se obtienen 4 casos, cada uno correspondiente a cada regla en la semántica de pasos cortos.
Los objetivos de prueba generados por las reglas \verb|Return_void| y \verb|Return_void_None| se resuelven automáticamente.
Los objetivos de prueba generados por las reglas \verb|Base| y \verb|None| se resuelven automáticamente luego de agregar los lemas~\ref{lemma:en_neg_by_pos} y ~\ref{lemma:tr_eval_upd_com}.
\end{proof}


Demostración del lema~\ref{lemma:small_step'_determ}:
\begin{lemmano}[small\_step'\_determ]
$\newline$
$s\ \rightarrow'\ s'\ \wedge\ s\ \rightarrow'\ s''\ \Longrightarrow\ s'\ =\ s''$
\end{lemmano}

\begin{proof}
La demostración es una prueba por casos sobre la semántica de pasos cortos.
Se demuestra automáticamente mediante el uso del lema~\ref{lemma:small_step_determ}.
\end{proof}
