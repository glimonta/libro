\chapter{Ejemplo de \textit{pretty printing}: Factorial}
\label{ap:factorial_ejemplo}
\lhead{Apéndice K. \emph{Ejemplo de \textit{pretty printing}: Factorial}}

Factorial se define de la siguiente manera en Isabelle:

\begin{lstlisting}[mathescape=true]
  definition factorial_decl :: fun_decl
    where "factorial_decl $\equiv$
      ( fun_decl.name = fact,
        fun_decl.params = [n],
        fun_decl.locals = [r, i],
        fun_decl.body =
          r ::= (Const 1);;
          i ::= (Const 1);;
          (WHILE (Less (V i) (Plus (V n) (Const 1))) DO
            (r ::= (Mult (V r) (V i));;
            i ::= (Plus (V i) (Const 1)))
          );;
          RETURN (V r)
      )"

  definition main_decl :: fun_decl
    where "main_decl $\equiv$
      ( fun_decl.name = main,
        fun_decl.params = [],
        fun_decl.locals = [],
        fun_decl.body =
          n ::= Const 5;;
          r ::= fact ([V n])
      )"

  definition p :: program
    where "p $\equiv$
      ( program.name = fact,
        program.globals = [n, r],
        program.procs = [factorial_decl, main_decl]
      )"
\end{lstlisting}


Mediante el uso del proceso de \textit{pretty printing} se genera el siguiente código en lenguaje C:

\begin{lstlisting}[mathescape=true]
  #include <stdlib.h>
  #include <stdio.h>
  #include <limits.h>
  #include <stdint.h>
  #include "../test_harness.h"
  #include "../malloc_lib.h"
  #ifndef INTPTR_MIN
    #error ("Macro INTPTR_MIN undefined")
  #endif
  #ifndef INTPTR_MAX
    #error ("Macro INTPTR_MAX undefined")
  #endif
  #if ( INTPTR_MIN + 1 != -9223372036854775807 )
    #error ("Assertion INTPTR_MIN + 1 == -9223372036854775807 failed")
  #endif
  #if ( INTPTR_MAX != 9223372036854775807 )
    #error ("Assertion INTPTR_MAX == 9223372036854775807 failed")
  #endif


  intptr_t n;
  intptr_t r;

  intptr_t fact(intptr_t n) {
    intptr_t r;
    intptr_t i;
    r = (1);
    i = (1);
    while ((i) < ((n) + (1))) {
      r = ((r) * (i));
      i = ((i) + (1));
    }
    return(r);
  }

  intptr_t main() {
    n = (5);
    (r) = (fact(n));
  }
\end{lstlisting}

Nótese que la verificación de los límites para los enteros para el límite inferior tiene una solución temporal ya que el valor de \verb|INTPTR_MIN| excede el límite superior para los enteros del preprocesador lo cual causa \textit{overflow} y una advertencia al compilar.~\footnote{Interpreta el número negativo como $-(numero)$ y produce una advertencia que indica que no puede representar $numero$.}
Para eliminar esta advertencia, se compara el valor de \verb|INTPTR_MIN + 1| al de \verb|INT_MIN + 1|.
