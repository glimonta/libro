\chapter{Proceso de Pruebas}\label{chapter:testing}
\lhead{Capítulo 5. \emph{Proceso de Pruebas}}

En los capítulos~\ref{chapter:semantics} y~\ref{chapter:pretty}, se formalizó la semántica para Chloe y se describió el proceso de traducción a código C.
Ahora se describe el proceso de pruebas hecho para incrementar la confianza en el proceso de traducción.
Se garantiza que el código generado de la semántica o bien termina con un error por falta de memoria o produce un resultado igual que el del mismo programa ejecutado en el ambiente Isabelle/HOL.
Se dice que los resultados de ejecutar el programa dentro de Isabelle y el programa generado en la máquina son iguales si los estados finales producidos por ambos son equivalentes.
Esto significa que se calcula el estado final producido por la ejecución de la semántica y se verifica que luego de ejecutar el programa generado el estado es equivalente, es decir, los contenidos de la memoria accesible y las variables globales son iguales.
Al generar programas se puede simplemente genera el código C por si solo o se puede incluir un conjunto de pruebas extras en forma de macros de C que garantizan la condición de equivalencia mencionada anteriormente.

En las siguientes secciones se describen los arneses de pruebas utilizados para generar las pruebas para el código.
Además se da una descripción mas detallada del significado que tiene que dos estados finales sean equivalentes.
Por último se habla un conjunto de pruebas y programas de ejemplos escritos en la semántica.
Sólo se genera código para programas válidos; esto es, se irá a un estado erróneo si surge algún comportamiento indefinido.
En este conjunto de pruebas se incluyen programas que se espera que alcancen un estado erróneo ya que presentan comportamiento indefinido o casos bordes.
Para estos programas \textit{incorrectos} no se generará código C.
También se presentan programas de ejemplo, tales como algoritmos de ordenamiento, para demostrar como funciona la semántica y el proceso de generación de código.


\section{Equivalencia de estados finales}

Se considera que un estado final producido por la ejecución de la semántica es igual al estado final de su programa generado en C cuando, al final de la ejecución, los valores de las variables globales son iguales para ambos casos y cada bloque de memoria accesible tiene contenidos iguales.
Se describe a continuación como se verifica esta equivalencia entre estados finales mediante el uso de pruebas.

\subsection{Generación de pruebas}

Se pueden generar pruebas para los programas.
Estas serán ejecutadas al final de la ejecución del programa entero y verificarán que el estado final del programa generado es el mismo que el estado final de la ejecución de la semántica.
Para comparar estos estados finales se comparan los valores de las variables globales al final de la ejecución con aquellos en el estado final de la ejecución de la semántica.

La dirección en la que las pruebas se hacen es tomando los valores del estado final de la ejecución de la semántica y verificando que la ejecución del código generado tenga los mismos valores que son esperados de acuerdo a la ejecución de la semántica.

Se introducen a continuación los distintos tipos de pruebas que se hacen dependiendo del contenido de una variable global.

\subsubsection*{Valores enteros}

Cuando el contenido de una variable global es un valor entero, simplemente se debe generar una prueba que compruebe que el valor entero en la variable global al final de la ejecución del código en C es el mismo que se obtiene de la valuación de la variable global en el estado final de la ejecución de la semántica.

\subsubsection*{Apuntadores}

Para verificar los apuntadores se tienen dos casos.
El caso del apuntador a \verb|NULL| y el caso donde el apuntador no es nulo.

Para valores \verb|NULL| se genera una prueba, similar a aquella para valores enteros, donde se comprueba que el contenido de la variable global es \verb|NULL|.

En el caso de apuntadores que tengan valor diferente a \verb|NULL|, se tiene un apuntador al bloque en memoria y se quiere verificar que el contenido del bloque en memoria, al final de la ejecución del programa generado, es igual al contenido del mismo bloque en el estado final en la semántica.
El bloque completo califica como memoria accesible, por lo que se debe verificar el contenido de cada celda en el bloque.
Para cada celda en el bloque se generan chequeos dependiendo de cual es el contenido que se espera que tenga la celda en memoria, es decir, un valor entero, un valor \verb|NULL| o un apuntador.

En el caso de chequeos para valores del tipo apuntador, se sigue cada apuntador hasta que se alcanza o bien un valor entero o un apuntador que ya fue seguido.
Al alcanzar un valor entero, se genera una prueba del tipo entero y al alcanzar un apuntador que ya fue seguido, se sabe que el camino no contiene apuntador alguno a memoria inválida.
Se verifica que la dirección del inicio del bloque es la misma que se obtiene al ajustar el apuntador que se siguió al inicio del bloque.
Para hacer esto, se deben seguir los apuntadores en un cierto orden y se debe mantener un conjunto de bloques de memoria ya \textit{visitados}.
De esta manera, al encontrar un apuntador a un bloque de memoria que ya fue chequeado (o \textit{visitado}) se puede parar y comparar las direcciones en lugar de seguir los apuntadores de manera cíclica indefinidamente.

\begin{comment}
Se presenta aquí la idea intuitiva detrás de la generación de pruebas y en las siguientes secciones se describen los detalles de implementación para los arneses de pruebas, ambos en Isabelle y C.
\end{comment}

\section{Arnés de prueba en Isabelle/HOL}

Primero, se define un nuevo tipo de datos por cada tipo de instrucción que se puede generar.
Esta definición se encuentra en la figura~\ref{fig:test_harness_datatype}.


\begin{figure}
\begin{lstlisting}[mathescape=true]
datatype test_instr =
  Discover string nat
| Assert_Eq string int_val
| Assert_Eq_Null string
| Assert_Eq_Ptr string nat

fun adjust_addr :: "int $\Rightarrow$ string $\Rightarrow$ string"
  where
  "adjust_addr ofs ca = shows_binop (shows ca) (''-'') (shows ofs) ''''"

definition ofs_addr :: "int $\Rightarrow$ string $\Rightarrow$ string"
  where
  "ofs_addr ofs ca =
    (shows ''*'' o
      shows_paren (shows_binop (shows ca) (''+'') (shows ofs))) ''''"

definition base_var_name :: "nat $\Rightarrow$ string" where
  "base_var_name i $\equiv$ ''__test_harness_x_'' @ show i"
\end{lstlisting}

\caption{Definiciones del arnés de prueba en Isabelle}
\label{fig:test_harness_datatype}
\end{figure}

Se tienen cuatro tipos de instrucciones de prueba diferentes que se pueden generar:

\begin{itemize}
  \item{\verb|Discover| representa la instrucción que agrega un bloque a la lista de bloques \textit{visitados}.
  El \verb|string| es la representación en cadena de caracteres de la expresión en C y el \verb|nat| representa el número de identificación del bloque de memoria actual.
  Las direcciones reales para los bloques de memoria asignada cambiarán con cada ejecución del programa en la máquina.
  La instrucción \verb|discover| empareja la dirección real para el inicio de un bloque con el número de bloque correspondiente en la representación abstracta.
  Para ello se generan variables locales con la función \verb|base_var_name| llamadas \verb|__test_harness_x_|$n$ donde $n$ es el número de identificación del bloque y en esas variables se guarda la dirección real del inicio del bloque para una ejecución en particular.}
  \item{\verb|Assert_Eq| representa la instrucción que chequea que el valor de una expresión sea el mismo valor entero que se espera que tenga.
  El \verb|string| es la representación en cadena de caracteres de la expresión en C y el \verb|int_val| representa el valor que se espera que esa expresión tenga de acuerdo al estado final en la ejecución de la semántica.}
  \item{\verb|Assert_Eq_Null| representa la instrucción que chequea que el valor de una expresión sea \verb|NULL|.
  El \verb|string| es la representación en cadena de caracteres de la expresión en C.}
  \item{\verb|Assert_Eq_Pointer| representa la instrucción que chequea que el valor de un apuntador apunte al mismo bloque al que se espera que apunte.
  El \verb|string| es la representación en cadena de caracteres de la expresión en C y el \verb|nat| representa el número de identificación del bloque de memoria.}
\end{itemize}

Se tienen también ciertas funciones auxiliares que asisten en el proceso de generación de pruebas.
La función \verb|adjust_addr| toma un desplazamiento y una representación en cadena de caracteres de una expresión en C (cuya evaluación resulta en un apuntador) y produce una representación en cadena de caracteres que ajusta la dirección al inicio del bloque al sustraer el desplazamiento de la misma.
La función \verb|ofs_addr| toma un desplazamiento y la representación en cadena de caracteres de una expresión en C (cuya evaluación resulta en un apuntador) y produce una representación en cadena de caracteres que ajusta la dirección para que apunte a la celda especificada por el desplazamiento al sumarle el mismo a la dirección.
Por último, la función \verb|base_var_name| dado un número natural $n$ produce la variable que se utiliza para el proceso de pruebas, la cual guarda la dirección que apunta al inicio del bloque $n$.
Esta variable siempre tendrá el prefijo \verb|__test_harness_x_| mas el número $n$.
Por ejemplo, para el número de bloque $2$ la función produce: ``\verb|__test_harness_x_2|''.


\begin{comment}
\begin{figure}
\begin{lstlisting}[mathescape=true]
context fixes $\mu$ :: mem begin

partial_function (option) dfs
  :: "nat set $\Rightarrow$ addr $\Rightarrow$ string $\Rightarrow$ (nat set $\times$ test_instr list) option"
  where
  [code]: "dfs D a ca = do {
    let (base,ofs) = a;

    case $\mu$!base of
      None $\Rightarrow$ Some (D,[])
    | Some b $\Rightarrow$ do {
        let ca = adjust_addr ofs ca;
        if base $\notin$ D then do {
          let D = insert base D;
          let emit = [Discover ca base];

          fold_option ($\lambda$i (D,emit). do {
            let i=int i;
            let cval = (ofs_addr i (base_var_name base));
            case b!!i of
              None $\Rightarrow$ Some (D,emit)
            | Some (I v) $\Rightarrow$ Some (D,emit @ [Assert_Eq cval v])
            | Some (NullVal) $\Rightarrow$ Some (D,emit @ [Assert_Eq_Null cval] )
            | Some (A addr) $\Rightarrow$ do {
                (D,emit') $\leftarrow$ dfs D addr cval;
                Some (D,emit@emit')
              }
          })
            [0..<length b]
            (D,emit)

        } else do {
          Some (D,[Assert_Eq_Ptr ca base])
        }
      }
  }"
end
\end{lstlisting}

\caption{DFS para generación de pruebas}
\label{fig:dfs_test}
\end{figure}
\end{comment}

Anteriormente se mencionó que los apuntadores deben ser seguidos hasta que se alcance o bien un entero o un valor \verb|NULL| o hasta que se alcance un apuntador que ya haya sido seguido.
Para hacer esto se deben seguir los apuntadores en un orden determinado y se deben mantener un conjunto de bloques ya \textit{visitados}.
De esta manera al encontrar un apuntadora a un bloque de memoria que ya fue chequeado (o \textit{visitado}) se puede parar y comparar los apuntadores en lugar de seguir los apuntadores de manera cíclica indefinidamente, chequeando una y otra vez partes de la memoria que ya fueron revisadas.
Los apuntadores se siguen en el orden dado por el algoritmo de búsqueda en profundidad (DFS)~\footnote{Por sus siglas en inglés \textit{Depth-First Search}}.
En el apéndice~\ref{ap:dfs_test} se presenta el algoritmo utilizado para seguir un apuntador y la explicación en detalle de su funcionamiento.
El algoritmo toma un conjunto de números naturales que son los bloques que ya han sido visitados, una dirección (la que se está siguiendo) y una representación en cadena de caracteres de la expresión en C (que debería contener la misma dirección).
Produce un nuevo conjunto de bloques visitados y una lista de instrucciones de prueba que se deben generar.

\begin{comment}
El algoritmo opera de la siguiente manera:
Primero, se intenta acceder al bloque, para ver si la memoria está libre o tiene contenido alguno.
Si la memoria está libre se retorna el mismo conjunto de bloques visitados y no se genera instrucción extra alguna.
Si la memoria no está libre, se ajusta la dirección al inicio del bloque, esto es porque al chequear la memoria queremos revisar los bloques completos ya que son parte de la memoria alcanzable.
Si el bloque que se está revisando pertenece ya al conjunto de visitados, simplemente se retorna el mismo conjunto de visitados y una instrucción \verb|Assert_Eq_Ptr| para chequear los apuntadores.
No obstante, si el bloque que se está revisando es un bloque que no ha sido \textit{visto} aun, se agrega al conjunto de visitados y se agrega una instrucción \verb|Discover| a la lista de instrucciones generadas.

Luego se chequea el contenido del bloque de memoria, comenzando por la primera celda hasta la última celda del bloque.
Al chequear cada celda se revisa si el contenido de la celda es un valor entero, \verb|NULL| o una dirección.
Si la celda contiene un valor entero, se retorna el mismo conjunto de visitados y se agrega una instrucción \verb|Assert_Eq| a la lista de instrucciones generadas.
Si la celda contiene un valor \verb|NULL|, se retorna el mismo conjunto de visitados y se agrega una instrucción \verb|Assert_Eq_Null| a la lista de instrucciones generadas.
Finalmente, si la celda contiene una dirección, se debe seguir esa dirección haciendo una llamada recursiva a \verb|dfs| antes de continuar revisando el bloque de memoria actual.
Al retornar de esta llamada, se retorna el nuevo conjunto de visitados resultante de la llamada recursiva y se agrega a la lista de instrucciones generadas hasta el momento la lista de instrucciones que retorna la llamada recursiva.
\end{comment}


\section{Arnés de prueba en C}

Con el fin de apoyar el proceso de pruebas en C se deben tener algunos macros que corresponden a aquellas instrucciones generadas en Isabelle.
Se crea un archivo de cabecera en C donde se definen los macros necesarios para realizar las pruebas así como algunas variables útiles.
Este archivo de cabecera se presenta en el apéndice~\ref{ap:header_test_harness}.
Allí se pueden encontrar las definiciones para los macros correspondientes a las instrucciones \verb|Discover|, \verb|Assert_Eq|, \verb|Assert_Eq_Null| y \verb|Assert_Eq_Ptr|.
Para guardar el conjunto de bloques visitados se utiliza un \textit{hash set}.
La implementación del \textit{hash set} fue hecha por Sergey Avseyev y se encuentra en línea~\citep{hashset}.
Se definen variables que contienen el número total de pruebas, el número de pruebas exitosas y fallidas y el \textit{hash set} de bloques visitados.
Las instrucciones de pruebas son traducidas mediante un proceso de \textit{pretty printing} que se detalla en el apéndice~\ref{ap:test_harness_c}.

\begin{comment}
\begin{figure}
\begin{lstlisting}[mathescape=true]
#include <stdlib.h>
#include <stdio.h>
#include <limits.h>
#include <inttypes.h>
#include "hashset.h"

hashset_t __test_harness_discovered;
int __test_harness_num_tests = 0;
int __test_harness_passed = 0;
int __test_harness_failed = 0;

#define __TEST_HARNESS_DISCOVER(addr, var)
  hashset_add(__test_harness_discovered, addr); var = addr;

#define __TEST_HARNESS_ASSERT_EQ(var, val)
  ++__test_harness_num_tests;
  (var != val) ? ++__test_harness_failed : ++__test_harness_passed;

#define __TEST_HARNESS_ASSERT_EQ_NULL(var)
  ++__test_harness_num_tests;
  (var != NULL) ? ++__test_harness_failed : ++__test_harness_passed;

#define __TEST_HARNESS_ASSERT_EQ_PTR(var, val)
  ++__test_harness_num_tests;
  (var != val) ? ++__test_harness_failed : ++__test_harness_passed;
\end{lstlisting}

\caption{Header file test\_harness.h}
\label{fig:header_test_harness}
\end{figure}
\end{comment}


\begin{comment}
Las instrucciones de pruebas son traducidas mediante \textit{pretty printing} a macros de C utilizando la función \verb|test_instructions|:

\begin{lstlisting}[mathescape=true, frame=single]
definition tests_instructions :: "test_instr list $\Rightarrow$ nat $\Rightarrow$ shows" where
  "tests_instructions l ind $\equiv$ foldr ($\lambda$
      (Discover ca i) $\Rightarrow$
        indent_basic ind
          (shows ''__TEST_HARNESS_DISCOVER '' o
            shows_paren
             ( shows ca o shows '', '' o shows (base_var_name i)))
    | (Assert_Eq ca v) $\Rightarrow$
        indent_basic ind
          (shows ''__TEST_HARNESS_ASSERT_EQ '' o
            shows_paren ( shows ca o shows '', '' o shows v))
    | (Assert_Eq_Null ca) $\Rightarrow$
        indent_basic ind
          (shows ''__TEST_HARNESS_ASSERT_EQ_NULL '' o
            shows_paren ( shows ca ))
    | (Assert_Eq_Ptr ca i) $\Rightarrow$
        indent_basic ind
          (shows ''__TEST_HARNESS_ASSERT_EQ_PTR '' o
            shows_paren
             ( shows ca o shows '', '' o shows (base_var_name i)))
    ) l"
\end{lstlisting}


Para cada bloque para el cual se genera una instrucción \verb|Discover| se debe también imprimir (\textit{pretty print}) una declaración para cada una de las variables utilizadas en las pruebas.
Esto se hace de la siguiente manera:

\begin{lstlisting}[mathescape=true, frame=single]
  definition tests_variables :: "test_instr list $\Rightarrow$ nat $\Rightarrow$ shows" where
    "tests_variables l ind $\equiv$ foldr ($\lambda$
      (Discover _ i) $\Rightarrow$
        indent_basic ind
          (shows dflt_type o shows '' *'' o shows (base_var_name i))
      | _ $\Rightarrow$ id
      ) l"
\end{lstlisting}


Por último, se puede obtener la lista de instrucciones que deben ser generadas usando la función \verb|emit_global_tests|.
Dada una lista de variables, genera una lista de instrucciones de prueba para las cuales se genera código en C.
Esta función se define como sigue:


\begin{lstlisting}[mathescape=true, frame=single]
definition
  emit_globals_tests ::
    "vname list $\Rightarrow$ state $\rightharpoonup$ (nat set $\times$ test_instr list)"
where "emit_globals_tests $\equiv$ $\lambda$vnames ($\sigma$,$\gamma$,$\mu$).
  fold_option ($\lambda$x (D,emit). do {
    case $\gamma$ x of
      Some vo $\Rightarrow$ do {
        let cai = x;
        case vo of
            None $\Rightarrow$ Some (D,emit)
          | Some (I v) $\Rightarrow$ Some (D,emit @ [Assert_Eq cai v])
          | Some (NullVal) $\Rightarrow$ Some (D,emit @ [Assert_Eq_Null cai] )
          | Some (A addr) $\Rightarrow$ do {
              (D,emit $\leftarrow$ dfs $\mu$ D addr cai;
              Some (D,emit@emit')
            }
      }
    | _ $\Rightarrow$ Some (D,emit)
  }
  ) vnames ({},[])"
\end{lstlisting}
\end{comment}


\section{Pruebas}

\subsection{Generación de código con pruebas}\label{subsection:codegen_with_tests}

En la sección~\ref{section:exporting_c_code} se describe una forma de exportar programas en C.
Existe una segunda manera de exportar programas en C donde además se genera código C para verificar la igualdad de los estados finales.

Anteriormente, se definió la manera en que cada construcción necesaria para las pruebas se imprime (\textit{pretty printing}), ahora se describe como se genera el código con las pruebas.

Se tiene una función similar a \verb|prepare_export| que prepara un programa para ser exportado con pruebas generadas, se define en la figura~\ref{fig:prepare_test_export} (donde $\Downarrow$ representa el carácter de salto de línea).
Primero, se obtiene el código del programa sin pruebas utilizando la función \verb|prepare_export|.
Solo se generan las pruebas para programas válidos cuya ejecución produzca un estado final, por lo tanto se debe verificar esto mediante la ejecución del programa.
Luego, se crea una lista de pruebas para las variables globales.
Más adelante, se genera la cadena de caracteres para las declaraciones de variables, una cadena de caracteres para inicializar el \textit{hash set} y la cadena de caracteres para las llamadas a los macros definidos en C.
Se tienen tres variables en el arnés de pruebas en C que llevan la cuenta de cuantas pruebas son ejecutadas, cuantas fallan y cuantas son exitosas.
Se quiere obtener algún tipo de información sobre el resultado de correr las pruebas generadas para el código.
Para ello se genera código que imprime a la salida estándar (luego de ejecutar el programa) los resultados del proceso de pruebas, es decir, el número de pruebas exitosas y fallidas.
Por último, \verb|prepare_test_export| produce una tupla que contiene el código para el programa sin pruebas y las pruebas generadas para el mismo.

\begin{figure}
\begin{lstlisting}[mathescape=true]
definition prepare_test_export :: "program $\Rightarrow$ (string $\times$ string) option"
where "prepare_test_export prog $\equiv$ do {
  code $\leftarrow$ prepare_export prog;
  s $\leftarrow$ execute prog;
  let vnames = program.globals prog;
  (_,tests) $\leftarrow$ emit_globals_tests vnames s;
  let vars = tests_variables tests 1 '''';
  let instrs = tests_instructions tests 1 '''';
  let failed_check = failed_check prog;
  let init_hash = init_disc;
  let nl = ''$\Downarrow$'';
  let test_code =
    nl @ vars @ nl @ init_hash @ nl @ instrs @ nl @ failed_check @ nl @ ''}'';
  Some (code,test_code)
}"
\end{lstlisting}

\caption{Función que prepara un programa para exportación con pruebas}
\label{fig:prepare_test_export}
\end{figure}


Con el fin de exportar el código con las pruebas agregadas, se define una nueva función en ML llamada \verb|generate_c_test_code|, su definición se encuentra en el apéndice~\ref{ap:generate_c_test_code}.
Esta función toma cuatro parámetros.
El primer parámetro de la función es de tipo \verb|(string, string) option| que corresponde a la tupla que contiene la representación en cadena de caracteres del programa en código C y las pruebas para ese programa, respectivamente.
Si la función recibe un valor \verb|None|, significa que la función \verb|prepare_test_export| falló y no se debe generar un programa en C.
El segundo parámetro es la ruta al directorio al que se quiere exportar el programa.
Este parámetro es una ruta relativa al directorio donde se encuentra la teoría de Isabelle que contiene las directivas para hacer \textit{pretty printing}.
El tercer parámetro es el nombre del programa.
El último parámetro es el contexto de la teoría actual.

Si la ruta dada es una cadena de caracteres vacía el código generado se imprimirá a la vista de salida de Isabelle.
Esta función crea un nuevo archivo con el nombre indicado en los parámetros con ``\verb|.c|'' agregado al final (es decir, $<nombre>\mathtt{.c}$) en el directorio indicado por el parámetro de la ruta.
El usuario puede entonces encontrar el código generado en el directorio indicado.

Esta función escribe el código C para un programa en un archivo y le agrega las pruebas generadas para el mismo al final de la función \verb|main|.

La función \verb|expect_failed_test| es muy similar a la función \verb|expect_failed_export| presentada en la sección~\ref{section:exporting_c_code} pero con un mensaje de error diferente.
Esta función genera un error en Isabelle si se genera código a partir del programa para el cual se esperaba una falla y no realizará acción alguna cuando no se genere código.

\begin{comment}
\begin{figure}
\begin{lstlisting}[mathescape=true]
  fun generate_c_test_code (SOME (code,test_code)) rel_path name thy =
    let
      val code = code |> String.implode
      val test_code = test_code |> String.implode
    in
      if rel_path="" orelse name="" then
        (writeln (code ^ " <rem last line> " ^ test_code); thy)
      else let
        val base_path = Resources.master_directory thy
        val rel_path = Path.explode rel_path
        val name_path = Path.basic name |> Path.ext "c"

        val abs_path = Path.appends [base_path, rel_path, name_path]
        val abs_path = Path.implode abs_path

        val _ = writeln ("Writing to file " ^ abs_path)

        val os = TextIO.openOut abs_path;
        val _ = TextIO.output (os, code);
        val _ = TextIO.flushOut os;
        val _ = TextIO.closeOut os;

        val _ = Isabelle_System.bash ("sed -i '$\mathtt{\$}$d ' " ^ abs_path);

        val os = TextIO.openAppend abs_path;
        val _ = TextIO.output (os, test_code);
        val _ = TextIO.flushOut os;
        val _ = TextIO.closeOut os;
      in thy end
    end

  | generate_c_test_code NONE _ _ _ =
      error "Invalid program or failed execution"

  fun expect_failed_test (SOME _) = error "Expected Failed test"
    | expect_failed_test NONE = ()
\end{lstlisting}

\caption{Generación de código en C con pruebas}
\label{fig:generate_c_test_code}
\end{figure}
\end{comment}


\subsection{Pruebas incorrectas}

Al desarrollar una \textit{suite} de pruebas es importante considerar casos incorrectos.
Se escribió un conjunto de programas para los cuales se espera que la generación de código falle.
Al utilizar las funciones \verb|expect_failed_export| y \verb|expect_failed_test| definidas en las secciones~\ref{section:exporting_c_code} y~\ref{subsection:codegen_with_tests}, respectivamente, se pueden escribir programas incorrectos y al generar código C para ellos se le puede indicar a Isabelle que debe esperar que esos procesos fallen de modo que no genere un error.

Tener programas incorrectos es muy útil ya que sirven como pruebas de regresión.
Al agregar nuevas características a la semántica se pueden correr todas las pruebas en la \textit{suite} de pruebas.
Si alguno de esos programas incorrectos se ejecuta de manera exitosa en la semántica y se genera código en C, se genera un error en Isabelle que indica que se está generando código C para programas que se espera que sean fallidos.
Estas pruebas son útiles para detectar errores si los cambios hechos alteran la semántica existente.

Sin embargo, es importante notar que para que una \textit{suite} de pruebas de regresión sea útil debe cubrir tantos casos como sea posible, lo cual, al trabajar con un lenguaje mas grande que Chloe, requiere una cantidad considerable de pruebas escritas.

\section{Programas de ejemplo}

Además de las pruebas descritas en esta sección se presenta un conjunto de programas de ejemplo en Chloe.
Estos tienen la finalidad de mostrar como se escriben programas en Chloe y como funcionan la ejecución y la generación de código.
Estos se encuentran en el código fuente presentado junto a este trabajo.

\begin{comment}
Los programas de ejemplo incluidos en el código fuente son:

\begin{itemize}
  \item{Bubblesort: implementación del algoritmo de ordenamiento bubblesort.}
  \item{Count: implementación de una función que cuenta las ocurrencias de un elemento en un arreglo.}
  \item{Cyclic linked list: implementación de una lista enlazada cíclica.}
  \item{Factorial: implementación de la función factorial.}
  \item{Fibonacci: implementación de una función que calcula el número Fibonacci de un número dado.}
  \item{Linked list: implementación de una lista enlazada.}
  \item{Mergesort: implementación del algoritmo de ordenamiento mergesort.}
  \item{Minimum: implementación de una función que retorna el elemento mínimo de un arreglo.}
  \item{Quicksort: implementación del algoritmo de ordenamiento quicksort.}
  \item{Selectionsort: implementación del algoritmo de ordenamiento selectionsort.}
  \item{String length: implementación de una función que calcula la longitud de una cadena que termina en cero.}
\end{itemize}
\end{comment}

\section{Ejecución de pruebas}

\subsection{Ejecución de pruebas en Isabelle}

En el código fuente presentado con este trabajo se tiene un directorio que contiene todas las pruebas y programas de ejemplo escritos para Chloe.
Es necesaria una forma de ejecutar esas pruebas en Isabelle automáticamente.

En el código fuente se incluye una teoría de Isabelle llamada ``\verb|All_Tests.thy|'' la cual es simplemente un archivo de Isabelle que importa cada prueba escrita para Isabelle, ambas correctas e incorrectas.
Al abrir este archivo en Isabelle, todos los archivos correspondientes a las pruebas y programas de ejemplo serán cargados.
Se tendrán entonces dos casos para cada prueba.

Para pruebas correctas y programas de ejemplos, se generará código.
Para pruebas incorrectas, no se generará código.
En el caso en el que ocurre un error, bien sea que se genere código para pruebas inválidas, o que el código no se genere para pruebas válidas, habrá un error.
Es posible ver estos de manera fácil en la vista llamada `Theories' de la interfaz gráfica de Isabelle ya que los archivos con un error son marcados en rojo.

\subsection{Ejecución de pruebas en C}

Cuando se genera código exitosamente, se desea compilar y correr las pruebas de manera automática.
Se tiene un \textit{Makefile} que compila cada prueba en la \textit{suite} así como también un \textit{shell script} de \textit{bash} que corre cada prueba en la \textit{suite}.
El resultado de ejecutar las pruebas será el número de pruebas exitosas y/o fallidas.
Cuando una prueba falla, el resultado se imprime en color rojo para hacerlo mas visible al usuario.
Además, otros comportamientos son atrapados por el \textit{script} y mostrados al usuario, tales como errores por falta de memoria, \textit{segmentation faults} y programas que terminen con alguno de los códigos de salida reservados~\citep{bash-scripting}.
Este \textit{script} se presenta en el apéndice~\ref{ap:bash_script}.

\begin{comment}
\begin{figure}
\begin{lstlisting}
#!/bin/bash

TEST_NAMES=(bubblesort_test count_test fact_test fib_test
  mergesort_test min_test occurs_test quicksort_test selection_test
  strlen_test plus_test subst_test outer_scope_test local_scope_test
  global_scope_test global_scope2_test mod_test div_test mult_test
  less_test and_test or_test not_test eq_test new_test deref_test
  while_test returnv_test linked_list_test cyclic_linked_list_test)

for test_name in ${TEST_NAMES[@]}
do
  res=$(./${test_name});
  ret=$?

if [[ ${res} == Failed* ]];
  then
    echo -e "\e[31mFAILED: \e[39m${res}"
else
  case ${ret} in

  1) echo -e "\e[31mError\e[39m (general error)
    occurred in the execution of ${test_name}";;
  2) echo -e "\e[31mError\e[39m (misuse of shell builtins)
    occurred in the execution of ${test_name}";;
  3) echo -e "\e[31mMemory allocation error\e[39m ocurred
    in execution of ${test_name}";;
  126) echo -e "\e[31mError\e[39m (command invoked cannot execute)
    occurred in the execution of ${test_name}";;
  127) echo -e "\e[31mError\e[39m (command not found)
    occurred in the execution of ${test_name}";;
  128) echo -e "\e[31mError\e[39m (invalid argument given to exit)
    occurred in the execution of ${test_name}";;
  130) echo -e "\e[31mError\e[39m (program terminated by Ctrl+C)
    occurred in the execution of ${test_name}";;
  139) echo -e "\e[31mSegmentation fault\e[39m ocurred
    in execution of ${test_name}";;
  *) echo ${res};;
  esac
fi
done
\end{lstlisting}

\caption{\textit{Shell script} para la ejecución de pruebas}
\label{fig:bash_script}
\end{figure}
\end{comment}

\section{Resultados}\label{section:results}

Tener una \textit{suite} de pruebas que permite verificar si el proceso de traducción es hecho correctamente fue muy valioso durante el desarrollo de este trabajo.
Originalmente, se tenían programas simples escritos en el lenguaje que ayudaban a verificar intuitivamente que el proceso de traducción se estaba haciendo de manera correcta y que la semántica del programa no estaba siendo cambiada.
Posteriormente, fue necesario automatizar el proceso de pruebas y diseñar pruebas más específicas.
Se procedió a agregar el arnés de pruebas y a crear la batería de pruebas.
Los resultados obtenidos de las pruebas fueron positivos.
Todos los casos incorrectos fallaron, como era esperado.
Para todos los casos correctos se genera código con pruebas y todas las pruebas generadas para estos programas son ejecutadas de manera exitosa.

Es importante notar que los casos de prueba fueron escritos por los mismos desarrolladores del proyecto, por lo que no se pueden establecer métricas exactas acerca de los resultados obtenidos durante el proceso de pruebas.
Es posible que algunos casos no estén contemplados ni cubiertos en la \textit{suite} de pruebas dado que las pruebas fueron escritas por los desarrolladores.
Es por ello que nuevos casos de prueba pueden, y deberían, ser agregados a la \textit{suite} de pruebas con el fin de continuar incrementando la confianza en el proceso de traducción.
Por otra parte, nuevas pruebas deben ser agregadas al agregar nueva funcionalidad a la semántica.
